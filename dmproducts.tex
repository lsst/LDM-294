\section{Products \label{sect:products}}

The products of DM are not the data products defined in \citeds{LSE-163}, rather they are the artifacts, systems and services,  we need to produce those products. \secref{sect:dmarc} outlines the highest level of this for dm while  \appref{sect:prodlist}  defines the complete product tree for DM and it is pictorially represented at a trimmed level in  \figref{fig:prods}. 
\citeds{LDM-148} provides a trace of products to requirements, while \appref{sect:prodlist} proves a full list with technical manager WBS element and product owner for each.  
Our primary guiding requirements come from \citeds{LSE-30} and \appref{sect:tracefor} and \appref{sect:traceback} provide the traceability to and from the OSS requirements tot he DM requirements \citeds{LSE-61}. 

\begin{figure}[htbp]
	\begin{center}
		 \includegraphics[height=19cm]{ProductTree}
		 \caption{DM product tree. \label{fig:prods}- there are over 200 products, this tree is to convey and idea of the products and is truncated to make it somewhat legible. 
		 The full list is given in \appref{sect:prodlist}
	 }
		  
	 \end{center}
 \end{figure}

\figref{fig:prods} 
 contains the WBS element associated with the component as weal as an git repositories belonging to them.
 Since the figure stops at level 3 most git repos will only be found in full list in\appref{sect:prodlist}.

 Every git repo should appear in \appref{sect:prodlist}  and hence have a technical manager and product owner identified. The table is hierarchical hence if the manager/owner is not filled in (or gone) we may go to the parent element manager/owner.

 Every JIRA component should map to one row in \appref{sect:prodlist} thus providing a contact for that component. 
