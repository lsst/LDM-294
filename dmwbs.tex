\section{WBS Elements } \label{sect:WBS}

DM WBS is fully defined in \citeds{LPM-43} with definitions provided in \citeds{LPM-44}. 
\citeds{LDM-472} defines the project controls for DM. The LSST Project Controller is Kevin Long. He is responsible for the PMCS and, in particular, for ensuring that DM properly complies with
our earned value management requirements. He is the first point of contact for
all questions about the PMCS system.


The WBS provides a hierarchical index of all hardware, software, services, and
other deliverables which are required to complete the LSST Project. It
consists of alphanumeric strings separated by periods. The first component is
always “1”, referring the LSST Construction Project. “02C” in the second
component corresponds to Data Management Construction. Subdivisions thereof
are indicated by further digits. Subdivisions at this level correspond to
teams within the DM project. The top level WBS elements are mapped to the lead institutes in \tabref{tab:wbs}.
The various groups involved in the WBS are briefly described in \secref{sect:groups}.

\begin{table}
\caption{DM top level Work Breakdown Structure \label{tab:wbs}}
\begin{tabular}[htb]{l|l|l}\\ \hline
{\bf WBS}    &  {\bf Description}   &                            {\bf Lead Institution}\\ \hline
	1.02C.01& System Management                       &  LSST\\ \hline
	1.02C.02& Systems Engineering                     &  LSST\\ \hline
	1.02C.03& Alert Production                        &  University of Washington\\ \hline
	1.02C.04& Data Release Production                 &  Princeton University\\ \hline
	1.02C.05& Science User Interface                  &  Caltech IPAC\\ \hline
	1.02C.06& Science Data Archive                    &  SLAC\\ \hline
	1.02C.07& Processing Control \& Site Infrastructure & NCSA\\ \hline
	1.02C.08& International Communications. \& Base Site& NCSA \& LSST\\ \hline
	1.02C.09& Systems Integration \& Test               & NCSA \& LSST\\ \hline
	1.02C.10& Science Quality \& Reliability Engineering& LSST\\ \hline
\end{tabular}
\end{table}


\section {Products} \label{sect:products}

The products of DM are not the data products defined in \citeds{LSE-163}, rather they are the artefacts, systems and Services,  we need to produce those produces. \figref{fig:prods} is an initial attempt to define the product tree for DM. 

\begin{figure}[htbp]
\begin{center}
 \includegraphics[height=19cm]{ProductTree}
\caption{DM product tree. \label{fig:prods}}
\end{center}
\end{figure}

