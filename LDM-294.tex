\documentclass[DM,toc]{lsstdoc}

\setDocChangeRecord{%
\addtohist{1.1}{2004-06-23}{Initial version (Document-139).}{J.~Kantor}
\addtohist{1.2}{2011-07-12}{Updated for PDR.}{J.~Kantor}
\addtohist{1.3}{2014-03-07}{Updated for construction phase.}{J.~Kantor}
\addtohist{1.4}{2014-10-21}{SQuaRE section added.}{J.~Kantor}
\addtohist{1.5}{2014-10-30}{Added LDM-294 handle.}{J.~Kantor}
\addtohist{2.0}{2015-03-11}{Updated with new RFC process, realignment of TCT, SAT, DMLT.}{J.~Kantor}
\addtohist{3.0}{2017-06-30}{Complete overhaul of content, with all new authors. Rewritten in LaTeX. Approved for release by W.~O'Mullane.}{W.~O'Mullane}
\addtohist{3.1}{2017-07-04}{Minor cleanups for review. Approved in \href{https://jira.lsstcorp.org/browse/RFC-358}{RFC-358}.}{W.~O'Mullane}
}

\title[DM PMP]{Data Management Organization and Management}

\author   {William O'Mullane, John Swinbank, Mario Juric and DMLT}
\setDocRef      {LDM-294} % the reference code
\setDocDate     {2017-07-04}              % the date of the issue
\setDocUpstreamLocation{\url{https://github.com/lsst/LDM-294}}

%
% a short abstract
%
\setDocAbstract {
This management plan covers the organization and management of the Data Management subsystem during the development, construction, and commissioning of LSST.
It sets out DM goals and lays out the management organization roles and responsibilities to achieve them.
It provides a high level overview of DM architecture, products and processes.
It provides a structured starting point for understanding DM and pointers to further documentation.}

\begin{document}
%
% the title page
%
\maketitle

%\printnoidxglossaries
%
% It's all yours from here on
%
\section{Introduction}
\subsection{Purpose}
This document defines the mission, goals and objectives, organization and responsibilities of the LSST Data Management (DM).  The document is currently scoped to define these elements for the LSST Design and Development, Construction, and Commissioning phases.  It does not address any ongoing mission for the DM during LSST operations.

\subsection{Mission statement}
Stand up operable, maintainable, quality services to deliver high-quality LSST data products for science and education, all on time and within reasonable cost.

\subsection{Goals And Objectives}
LSST Data Management will:
\begin{itemize}
\item Define the data products, data access mechanisms, and data management and curation requirements for the LSST
\item Assess current and LSST-time frame technologies for use in providing engineered solutions to the requirements
\item Define the computing, communications, and storage infrastructure and services architecture underlying LSST data management
\item Select, implement, construct, test, document, and deploy the LSST data management infrastructure, middleware, applications, and external interfaces
\item Document the operational procedures associated with using and maintaining the LSST data management capabilities
\item Evaluate, select, recruit, hire/contract and direct permanent staff, contract, and in-kind resources in LSST and from partner organisations participating in LSST Data Management initiatives.

\end{itemize}


The DM goals in selecting and, where necessary, developing LSST software solutions are:

\begin{itemize}
	\item Acquire and/or develop solutions: To achieve its mission, LSST DM subsystem prefers to acquire and configure existing, off-the-shelf, solutions. Where no satisfactory off-the-shelf solutions are available, DM develops the software and hardware systems necessary to:
\begin{itemize}
	\item Enable the generation of LSST data products at the LSST Archive and Satellite processing center, and
	\item Enable the the serving of LSST data products from the two LSST DACs (one in the U.S., and one in Chile).
\end{itemize}
	\item Maintain coherent architecture: DM software architecture is actively managed at the subsystem level. A well engineered, and cleanly designed codebase is less buggy, more maintainable, and makes developers who work on it more productive. Where there is no significant impact on capabilities, budget, or schedule, LSST DM prefers to acquire and/or develop reusable, open source, solutions.
	\item Support reproducibility and insight into algorithms: Other than when prohibited by licensing, security, or other similar considerations, DM makes all newly developed source code public, especially the Science Pipelines code. Our primary goal in publicising the code is to simplify reproducibility of LSST data products, and provide insight into algorithms used. The software is to be documented to achieve those goals. 
	\item Opportunities beyond LSST: LSST DM codes may be of interest and (re)used beyond the LSST project (e.g., by other survey projects, or individual LSST end-users). While enabling or supporting such applications goes beyond LSST’s construction requirements, cost and schedule-neutral technical and programmatic options that do not preclude them and allow for future generalisation should be strongly preferred.


\end{itemize}

Background decision material on choices made in DM will be documented in technical notes (DMTN) which will be lodged in DocuShare (see \secref{sect:docman}).

\section{Data Management Conceptual Architecture \label{sect:dmarc}}

The DM Subsystem Architecture is detailed in \citeds{LDM-148}.
A few of the higher level diagrams are reproduced here to orient the reader within DM.

During Operations, components of the DM Subsystem will be installed and run in
multiple locations. These include:

\begin{itemize}
\item The Commissioning Cluster, which may be physically at NCSA in Urbana-Champaign
\item The main compute enclave, also at NCSA in Urbana-Champaign
\item The US Data Access Center (DAC), also at NCSA in Urbana-Champaign
\item The Chilean DAC in the Base Facility in La Serena Chile
\item The Satellite Processing Center at CC-IN2P3 in Lyon, France
\end{itemize}

\figref{fig:dmsdeploy} shows the various DM components which will be used in operations and the physical compute environments in which they will be deployed.
Bulk data storage and transport between components is provided by the Data Backbone. This complex piece of infrastructure is displayed in \figref{fig:databb}.

Science users will access the data products produced by LSST through the
Science Platform, as shown in \figref{fig:sciplat}.

\figref{fig:dcs} shows the common infrastructure and services layer which underlies the compute environments.
This does not list specific technologies for management/monitoring, provisioning/deployment, or workload/workflow --- these are still under development --- but consider industry-standard tools such as Nagios, Puppet/vSphere/OpenStack/Kubernetes, and Pegasus.

\begin{figure}[htbp]
\begin{center}
\includegraphics[width=0.8\textwidth]{images/DMSDeployment}
\caption{DM components as deployed during Operations. Where components are
deployed in multiple locations, the connections between them are labeled with
the relevant communication protocols. Science payloads are shown in blue.
\label{fig:dmsdeploy}}
\end{center}
\end{figure}

\begin{figure}[htbp]
\begin{center}
\includegraphics[width=0.5\textwidth]{images/SciencePlatform}
\caption{The sub-components of the Science Platform. \label{fig:sciplat}}
\end{center}
\end{figure}


\begin{figure}[htbp]
\begin{center}
\includegraphics[width=0.6\textwidth]{images/DataBackbone}
\caption{The Data Backbone links all the physical components of DM. \label{fig:databb}}
\end{center}
\end{figure}

\begin{figure}[htbp]
\begin{center}
 \includegraphics[width=0.7\textwidth]{images/DMSCommonServices}
\caption{Common infrastructure services available at each DM location. \label{fig:dcs}}
\end{center}
\end{figure}



\subsection{External Interfaces \& Auxiliary Data}
The DM external interfaces are controlled by the ICDs listed in \tabref{tab:icds}.

\begin{table}
    \begin{center}
      \caption{DM Interface Control Documents \label{tab:icds}}
      \begin{tabular}{l p{0.7\textwidth}}
          \hline
          \citeds{LSE-68} & Data Acquisition Interface between Data Management and Camera\\
          \citeds{LSE-69} & Interface between the Camera and Data Management   \\
          \citeds{LSE-72} & OCS Command Dictionary for Data Management\\
          \citeds{LSE-75} & Control System Interfaces between the Telescope and Data Management\\
          \citeds{LSE-76} & Infrastructure Interfaces between Summit Facility and Data Management\\
          \citeds{LSE-77} & Infrastructure Interfaces between Base Facility and Data Management\\
          \citeds{LSE-130} & List of Data Items to be Exchanged Between the Camera and Data Management\\
          \citeds{LSE-131} & Data Management Interface Requirements to Support Education and Public Outreach \\
          \citeds{LSE-140} & Auxiliary Instrumentation Interface between Data Management and Telescope\\
          \hline
      \end{tabular}
    \end{center}
\end{table}

In addition, certain tasks in DM rely on external catalogs and other information.
The current design requires:
\begin{itemize}
\item Gaia catalog (Release 2) as a photometry baseline.
\end{itemize}

\section{Data Management Organization Structure}

This section defines the organization structure for the period in which the DM System is developed and commissioned, up to the start of LSST Observatory operations.

The DM Project Manager (William O'Mullane), Deputy Project Manger (John Swinbank) and DM Project Scientist (Mario Juri\'c), who are known collectively as DM Management, lead the DM Subsystem.
The Project Manager has direct responsibility for coordination with the overall LSST Project Office, the LSST Change Control Board, the LSST Corporation, and LSST partner organizations on all budgetary, schedule, and resource matters.
The Project Scientist has primary scientific and technical responsibility in the DM and responsibility for ensuring that the scientific requirements of the LSST are supported, and is a member on the LSST Project Science Team (PST).

As shown in \figref{fig:dmorg}, the organization now features  major products  each with a product owner
relating to a major element of the DM Subsystem (Level 2 Work Breakdown Structure elements).

\begin{figure}[htbp]
\begin{center}
 \includegraphics[width=\textwidth]{images/DmOrg}
\caption{DM organization with Scientists in Green. \label{fig:dmorg}}
\end{center}
\end{figure}
%figure wom


\subsection {Meetings } \label{sect:meetings}
As a diverse and distributed organization DM staff will participate in a considerable number of meetings.
NSF and Aura have many rules on meeting attendance and LSST keep policies updated accordingly in \citeds{LPM-191} and \citeds{Document-13760}. This includes the travel summary report template \citedsp{Document-13762} every traveler must fill after attending a meeting.

A detailed debrief note or presentation may be asked of travelers to specific meetings of interest by the DMLT.

\subsection {Working Groups } \label{sect:wgs}
Some issues in development of a system like Data Management require more effort to remove than a simple RFC. When t
he decision making process (\appref{sect:ddmp}) can not come to a conclusion the DM PM reserves the right to create
 a short lived working group to deal with the issue. A working group will be given a specific narrow charge, it will be a small group ($\approx 7$ people), it will be time bounded and have a clear deliverable. 
 Members of the group will be agreed by the DMLT to provide the best technical input from all stakeholders perspectives. Members of the working group should discuss in their local organizations and socialize recommendations ahead of adoption. 
 This has been done for the SuperTask for example. 

 \subsection {Studies } \label{sect:studies}
 In some cases DM will initiate studies by external parties to investigate potential alternatives this is especially
  true for technology related activities. 



\subsection {Document Management} \label{sect:docman}

DM documents will follow the Systems Engineering Guidelines of LSST. PDF versions of released documents shall be put in Docushare in accordance with the Project's Document Management Plan \citedsp{LPM-51}. LPM level documents are released on agreement of the DMCCB (\secref{sect:dmccb}), uncontrolled documents such as technotes may be released when the author decides it is appropriate or they are asked to release it by the Project Manager.

The Document Tree for DM is shown in \figref{fig:doctree}, it is not exhaustive but gives a high level orientation for the main documents in DM and how they relate to each other. Some documents shown in red are not yet written.

\begin{figure}
\begin{center}
 \includegraphics[width=0.9\textwidth]{images/DocTree}
\caption{Outline of the documentation tree for DM software relating the high level documents to each other. \label{fig:doctree}}
\end{center}
\end{figure}

\figref{fig:doctree} has one box for End User documentation, this is a major set of documentation for DM which will be web based as  described in  \citeds{LDM-493}. \figref{fig:eudoc} shows the intended web hierarchy for the end user documentation.

\begin{figure}
\begin{center}
 \includegraphics[width=1\textwidth]{images/EndUserDocs}
\caption{Outline of the web hierarchy for the DM end user documentation. \label{fig:eudoc}}
\end{center}
\end{figure}
%Figure from jsick



Service-level documentation follows the layered service architecture of the LSST Data Facility (see \figref{fig:servdoc}).

\begin{figure}
\begin{center}
 \includegraphics[width=0.5\textwidth]{images/servdocs}
\caption{Outline of layered service architecture of the Data Facility. \label{fig:servdoc}}
\end{center}
\end{figure}

\subsubsection {Documentation of Cross-Cutting Aspects for services}

The cross-cutting aspects of the LSST Data Facility, Security and Operational Manageability, are represented by the vertical boxes. Documentation of these aspects describes policies, procedures, and supporting management frameworks, including:
\begin{enumerate}
	\item	LDF service management framework: service catalog, service-level agreements (SLAs), configuration management database (CMDB), service monitoring.
	\item	LDF service management processes and context in the overall project: incident response, request response, issue tracking, problem management and the problem management database, change management and change control authority, release management.
	\item	Overview of the security enclave structure
	\item	Security controls and incident response procedures
	\item	Disaster recovery and continuity policies
\end{enumerate}

\subsubsection{Documentation of Service Layers}

The box at the top of the figure, Use Cases, represents subsystem-level and project-level operational use cases. The next layer, LDF-offered Services, represents specific services offered by the Data Facility which satisfy those use cases. Documentation of this layer includes:

\begin{enumerate}
\item	For each service, a Concept of Operations (ConOps) which summarizes how a service operates to satisfy a use case. The ConOps describes the operational characteristics of the production system, context within overall LSST operations, and representative scenarios. 
\item	For each service, a Theory of Operations, which provides a mental model of a constructed system. The Theory of Operations explains how the constructed service both fulfills the ConOps and integrates with the cross-cutting aspects of the facility. The document describes the overall architecture of the service and dependency on supporting service layers; integration into aspects of computer security, information security and business continuity; and integration into incident reporting and response, availability and capacity management, and change management.
\end{enumerate}

The next two layers, Reusable Production Services and Data, Compute, and IT Security Services, represent tiers of supporting service. Documentation of these layers includes a Theory of Operations, as described above, explaining the dependencies on supporting service and ITC layers, and integration with cross-cutting aspects of the facility.

The ITC box represents hardware components supporting all LDF services. Documentation of ITC describes the system elements at all facility sites, administration within each security enclave and integration with security operations, the overall provisioning plan, ITC system monitoring and integration into the service monitoring framework, and integration into service management processes including configuration management and change management.

The Software box represents service software components being developed by the LSST Data Facility. Documentation of software elements follows the standards of the LSST software stack.

Documents are managed as configuration items in the LSST Data Facility CMDB.

\subsubsection{Draft Documents}

Draft DM documents will be kept in GitHub. A single repository per document will be maintained with the head revision containing the \emph{released } version which should match the version on docushare. Each repository will be included as a \emph{submodule} of a single git repository located at \url{https://github.com/lsst-dm/dm-docs}.

Use of Google Docs or confluence is tolerated but final delivered documents must conform to the standard LSST format, and hence either produced with LaTeX, using the lsst-texmf package\footnote{\url{https://lsst-texmf.lsst.io}}, or Word, using the appropriate LSST template \citedsp{Document-9224, Document-11920}. The precursor document should then be erased with a pointer to the baseline document, stored in GitHub.

End user documentation will most likely and appropriately be web based and the scheme for that is described in \citeds{LDM-493}.

\subsection {Configuration Control} \label{sect:config}

Configuration control of documents is dealt with in \secref{sect:docman}. Here we consider more the operational systems and software configuration control.

\subsubsection{Software Configuration Control}

DM follows a git based versioning system based  on public git repositories and the approach is covered in the developer guide \url{https://developer.lsst.io/}.
The master branch is the stable code with development done in \emph{ticket} branches (named with the id of the corresponding JIRA Ticket describing the work.
Once reviewed a branch is merged to master.\footnote{LSE-14 seem out of date and should be updated or revoked - titled a guideline it seems inappropriate as an LSE.}

As we approach commissioning and operations DM will have a much stricter configuration control.
At this point there will be a version of the software which may need urgent patching, a next candidate release version of the software, and the master.
A patch to the operational version will require the same fix to be made in the two other versions.
The role of the DM Change Control Board (DMCCB; \secref{sect:dmccb}) becomes very important at this point to ensure only essential fixes make it to the live system as patches and that required features are included in planned releases.

We cannot escape the fact that we  will have multiple code branches to maintain in operations which will lead to an increase in work load.
Hence one should consider that perhaps more manpower may be needed in commissioning to cope with urgent software fixes while continuing development.
The other consideration would be that features to be developed post commissioning will probably be delayed more than one may think, as maintenance will take priority.\footnote{WOM identifies this as the maintenance surge.}

\subsubsection{Hardware Configuration Control}

On the hardware side we have multiple configurable items, we need to control which versions of software are on which machines. These days tooling like Puppet make this reasonably painless. Still the configuration  must be carefully controlled to ensure reproducible deployments providing correct and reproducible results. The exact set of released software and other tools on each system should be held in a configuration item list.
Changes to the configuration should be endorsed by the DMCCB.

The sizing model for compute hardware purchasing is detailed in \citeds{LDM-144} \citeds{LDM-141} and \citeds{LDM-138}.

\subsection {Risk Management } \label{sect:risk}

Risks will be dealt with within the LSST Project framework as defined in \citeds{LPM-20}.
Risks in DM may be sent to the DM project manger or Deputy project manager at any time for consideration to be included in the formal risk register (appropriate costed and weighted). All risks are reviewed regularly by the DM Project manager and Systems Engineer (minimum each 3 months).


\subsection {Quality Assurance  } \label{sect:pa}

In accordance with the project QA plan \citeds{LPM-55} we will perform QA on the software products.
This work will mainly be carried out by SQuaRE (\secref{sect:square}).
Quality assurance here means compliance with project guidelines for production, in out case of software production.
A part of this is to have a verification/validation plan(s) which in and of itself is a major task (see \secref{sect:vanv}).


\subsection{Action item control}
Actions in DM are tracked as JIRA issues an periodically reviewed at DMLT meetings.


\subsection {Verification and Validation } \label{sect:vanv}

We intend to verify and validate as much of DM as we can before commissioning and operations.
This will be achieved through testing and operations rehearsals/data challenges.
The verification and validation approach is detailed in \citeds{LDM-503} including a high level test schedule,
the top level schedule is given in \figref{fig:schedule}.

\newpage
\section{Project Controls}\label{sect:dmpc}

DM follows the LSST project controls system, as described in \citeds{LPM-98}.
Specific DM processes for project planning are elaborated in \citeds{LDM-472}.

The LSST Project Controller is responsible for the PMCS and, in particular, for ensuring that DM properly complies with our earned value management requirements.
He is the first point of contact for all questions about the PMCS system.

\subsection{Schedule  \label{sect:schedule} }
The entire LSST project schedule is held in Primavera. Tied to major project milestones we have  a series of DM tests which need to be performed to show readiness for he different project phases.
This is depicted in \figref{fig:schedule}.

\begin{figure}[htbp]
	\begin{center}
		 \includegraphics[width=\textwidth]{images/DMMasterSchedule}
		 \caption{DM major milestones(LDM-503-x) in the LSST schedule. \label{fig:schedule}}
	 \end{center}
 \end{figure}



\subsection{Work Breakdown Structure} \label{sect:WBS}

The DM WBS is laid out in \citeds{LPM-43} with definitions provided in \citeds{LPM-44},
the new WBS is currently described in \appref{sec:wbslist} making LPM-43 out of date.

The WBS provides a hierarchical index of all hardware, software, services, and other deliverables which are required to complete the LSST Project.
It consists of alphanumeric strings separated by periods.
The first component is always “1”, referring the LSST Construction Project.
``02C'' in the second component corresponds to Data Management Construction.
Subdivisions thereof are indicated by further digits.
These subdivisions correspond to teams within the DM project.
The top level WBS elements are mapped to the lead institutes in \tabref{tab:wbs}, the lead institutions roles are outlined in \secref{sect:leadtutes}.
The various groups involved in the WBS are briefly described in \secref{sect:groups}.

\begin{table}
\caption{DM top level Work Breakdown Structure \label{tab:wbs}}
\begin{tabular}[htb]{|l|l|l|} \hline
\textbf{WBS}  &  \textbf{Description}   &  \textbf{Lead Institution}\\ \hline
1.02C.01& System Management                       &  LSST Tucson \\ \hline
1.02C.02& Systems Engineering                     &  LSST Tucson \\ \hline
1.02C.03& Alert Production                        &  University of Washington\\ \hline
1.02C.04& Data Release Production                 &  Princeton University\\ \hline
1.02C.05& Science User Interface and Tools        &  Caltech IPAC\\ \hline
1.02C.06& Science Data Archive                    &  SLAC\\ \hline
1.02C.07& Processing Control \& Site Infrastructure & NCSA\\ \hline
1.02C.08& International Communications. \& Base Site&  LSST Tucson \\ \hline
1.02C.09& Systems Integration \& Test               & NCSA \& LSST Tucson \\ \hline
1.02C.10& Science Quality \& Reliability Engineering& LSST Tucson \\ \hline
\end{tabular}
\end{table}

\section{Products \label{sect:products}}

The products of DM are not the data products defined in \citeds{LSE-163},
but rather the artifacts, systems, and services which will be used by the
operational LSST system to generate those data products.

In \secref{sect:dmarc}, we briefly described the high level approach being taken to the design of the DM products, while \appref{sect:prodlist} provides a complete list of products, including the technical manager, WBS element, and product owner for each.
That information is summarized in the product tree shown in \figref{fig:prods}.

Each DM product is being developed to satisfy one or more of the requirements placed upon the DM subsystem. \citeds{LDM-148} provides a tracing from each product to and from the relevant requirements.
These requirements are drawn from \citeds{LSE-61}, the DM System Requirments document.
The requirements \citeds{LSE-61} are themselves traced to higher level requirements in the LSST System Requirements (LSR; \citeds{LSE-29}) and/or the Observatory System Specifications (OSS; \citeds{LSE-30}).
\appref{sect:tracefor} traces DM requirements to higher level requirements, and \appref{sect:traceback} tracing relevant  higher-level requirements to DM..

\begin{figure}[htbp]
	\begin{center}
		 \includegraphics[height=19cm]{ProductTree}
         \caption{An overview of the DM product tree. This provides just a summary of the highest level items: refer to \appref{sect:prodlist} for the full list.}
         \label{fig:prods}
	 \end{center}
 \end{figure}

Every code repository used by DM must be associated with a product, and hence will have an associated technical manager and product owner.

\section{Roles in Data Management}

This section describes the responsibilities associated with the roles shown in
\figref{fig:dmorg}.


\subsection{DM Project Manager (DMPM)\label{role:dmpm}}

The DM Project Manager is responsible for the efficient coordination of all LSST activities and responsibilities assigned to the Data Management Subsystem. The DM Project Manager has the responsibility of establishing the organization, resources, and work assignments to provide DM solutions.  The DM Project Manager serves as the DM representative in the LSST Project Office and in that role is responsible for presenting DM initiative status and submitting new DM initiatives to be considered for approval. Ultimately, the DM Project Manager, in conjunction with his / her peer Project Managers (Telescope, Camera), is responsible for delivering an integrated LSST system. The DM Project Manager reports to the LSST Project Manager. Specific responsibilities include:

\begin{itemize}
\item Manage the overall DM System
\item Define scope and funding for DM System
\item Develop and implement the DM project management and control process, including earned value management
\item Approve the DM Work Breakdown Structure (WBS), budgets and resource estimates
\item Approve or execute as appropriate all DM outsourcing contracts
\item Convene and/or participate in all DM reviews
\item Co-chair the DM Leadership Team (\ref{sect:dmlt})
\end{itemize}

\subsection{DM Deputy Project Manager (DDMPM) \label{role:dmdpm}}

The PM and deputy will work together on the general management of DM and any specific PM tasks may be delegated to the deputy as needed and agreed. In the absence of the PM the deputy carries full authority and decision making powers of the PM. The DM Project Manager will keep the Deputy Project Manager informed of all DM situations such that the deputy may effectively act in place of the Project Manager when absent.

\subsection{DM Subsystem Scientist (DMSS) \label{role:dmps}}

The DM Project Scientist has ultimate responsibility for ensuring DM initiatives provide solutions that meet the overall LSST scientific and technical requirements.  The DM Project Scientist must ensure correct specification of DM Scientific Requirements and proper translation of those requirements into derived information technology requirements and ultimately, into implemented solutions. The DM Project Scientist must ensure that the DM subsystem is properly scoped and integrated within the overall LSST system. The DM Project Scientist is also a member of the LSST Project Science Team (PST) and reports to the LSST Director. Specific responsibilities include:

\begin{itemize}
\item Ensuring that the scientific goals of the DM system are met
\item Set requirements for the DM that:
\begin{itemize}
\item Ensure that the design and operational flow of the data products meet the needs of the science community
\item Ensure that the quality requirements of the data products will be / are being met by the DMS, with a particular emphasis on choice of appropriate application algorithms
\end{itemize}
\item Set requirements for and assess/validate the results of Data Challenges and other precursor experiments
\item Set requirements and assess/validate results for Data Releases
\item Convene and/or participate in all DM reviews
\item Co-Chair the DM Leadership Team (\secref{sec:dmlt})
\item Chair the DM Subsystem Science Team (\secref{sec:dmsst})

\end{itemize}

\subsection{Project Controller/Scheduler \label{role:pcon}}

The DM Project Controller is responsible for integrating DM's agile planning process with the LSST Project Management and Control System (PMCS). Specific responsibilities include:

\begin{itemize}

  \item{Assist T/CAMs in developing the DM plan}
  \item{Synchronize the DM plan, managed as per \citeds{LDM-472}, with the LSST PMCS}
  \item{Ensure that the plan is kept up-to-date and milestones are properly tracked}
  \item{Coordinate sprints, ensuring that appropriate story points are allocated and tracked}
  \item{Create reports, Gantt charts and figures as requested by the DMPM}

\end{itemize}

\subsection{Technical/Control Account Manager (T/CAM) \label{role:tcam}}

Technical/Control Account Managers have managerial and financial responsibility
for the engineering teams within DM. Each T/CAM is responsible for a specific set of WBS elements. Their detailed responsibilities include:

\begin{itemize}

  \item{Developing, resource loading and maintaining the plan for executing the DM construction project within their WBS}
  \item{Synchronizing the constuction schedule with development in WBS elements managed by other T/CAMs}
  \item{Maintaining the budget for their WBS and ensuring that all work undertaken is charged to the correct accounts}
  \item{Working with the relevant Science Leads and Product Owners (\secref{role:prodo}) to devleop the detailed plan for each cycle and sprint as the occur}
  \item{Work with the DM Project Controller (\secref{role:pcon}) to ensure that all plans and milestones are captured in the LSST Project Controls system}
  \item{Perform day-to-day management of staff within their WBS}
  \item{Perform the role of ``scrum-master'' during agile development}

\end{itemize}

\subsection{Product Owner \label{role:prodo}}

A product owner is responsible for the quality and acceptance of a particular product.
The product owner should sign off on the requirements to be fulfilled in every delivery and therefore also on any descopes or enhancements.
The product owner should define tests which can be run to prove a delivery meets the requirements due for that product.

\subsection{Institutional Science Leads \label{role:scilead}}

The institutional science leads serve as product owners (\secref{role:prodo}) for the major components of the DM System (Alert Production, Data Release Production, Science User Interface etc).

In addition, they provide scientific and technical expertise to their local engineering teams.

They work with the T/CAM who has managerial responsibility for their product to define the overall construction plan and the detailed cycle plans for DM.

Institutional science leads are members of the DM System Science Team (\secref{sect:dmsst}) and, as such, report to the DM Subsystem Scientist (\secref{sect:dmps}).

\subsection{Pipeline Scientist \label{role:pipe}}

Several DM products come together to form the LSST pipeline. The Pipeline Scientist is the product owner for the overall pipeline. 
The Pipeline Scientist should :
\begin{itemize}
\item  Provide guidance and test criteria for the full pipeline including how QA is done on the products.  
\item Keep the big picture of where the codes are going in view. Predominantly the algorithms, but also the implementation and architecture (as part of the System Engineering Team \secref{sect:sysengt}).


\item Advise on how we should attack algorithmic problems,  
providing continuing advice to subsystem product owners as we try new things. 

\item Advise on calibration issues, provide understanding of the detectors from a DM point of view. 

\item Advise on the overall (scientific) performance of the system, and how we'll test it.  Thinking about all the small things that we have to get right to make the overall system good.



\end{itemize}


\subsection{Systems Engineer \label{role:sysengineer}}
With the system engineering team \secref{sect:sysengt} 
 the System engineer 
 owns the DM entries in the risk register and is generally in charge of the {\em process} of building DM products. 

The Systems engineer  is responsible for the requirements work:
\begin{itemize}
\item E.g., updating the DMSR, OSS, LSR (including traceability)
\item Ensure we’re appropriately modelling and recording information about the system (e.g.,
    drawings, design documents, etc.)
\item Overseeing work on ICDs, lower level requirements documents, etc.
\item Ensuring we have a solid verification plans/standards across the board in DM
\end{itemize}

The Systems Engineer is responsible for the process to define \& maintain DM interfaces (internal and external) 
\begin{itemize}
\item Defining standards for and ensuring internal interfaces are identified and worked out
\item Direct Interface Scientist's work on external ICDs
\end{itemize}

The Systems Engineer shall Chair the DM Change Control Board \secref{sect:dmccb}
\begin{itemize}
\item Organise DMCCB  processes so our change control process runs smoothly
\item Shepherd RFCs through change control
\item Monitor + Flag RFCs requiring DMCCB  attention
\item Call up meetings, make sure decisions are made, and recorded
\end{itemize}

The System Engineer represents DM on the LSST CCB.

\subsection{DM Interface Scientist (DMIS) \label{role:dmis}}
Look after all internal and external DM interfaces - including defining tests for them. 

\subsection{Requirements Engineer \label{role:reqeng}}
With the system engineering team (\secref{sect:sysengt}) and in close coordination with the software architect (\secref{role:softarc}) and the system engineer (\secref{role:sysengineer}) looks after the baseline requirements for DM.. 




\subsection{Software Architect \label{role:softarc}}
The software architect looks after the software we are building. How does it all fit together are their techniques/technologies we should be using. How can we minimise dependencies. 

With the \secref{role:sysengineer} the Software architect should also agree how to track requirements to code and verify requirements are i.e. are hooks required in the code ?

\subsection{Operations Architect \label{role:opsarc}}
{\em Margaret  or Don perhaps some text here .. }
The DM Operations  Architect is responsible for ensuring that all elements of the DM systems, including operations teams, infrastructure, middle ware, applications, and interfaces, 
all come together to form an operable system. 
Specific responsibilities include:
\begin{itemize}
\item Setting up and coordinating  Operations Rehearsals
\item Ensuring Readiness of procedures and personnel for Operations
\item Set standards for operations e.g. procedure handling and operator logging
\item Participate in stakeholder and end user coordination and approval processes and reviews
\item Member of the LSST System Engineering Team
\end{itemize}


\subsection{Configuration Manager (CM)}\label{role:cm}

%A configuration management role exists both at DM and Data Facility levels this could be the same individual
The DM Configuration Manager (CM) is responsible 
for Configuration Management activities inside DM and NCSA(?).
The following list is not exhaustive, but is intended as a guideline to the CM activities:
\begin{itemize}
	\item assure that Configuration Management Plan (CMP) is correctly applied and provide appropriate reasons in case of non conformance's 
 \item define which Configuration Items are to be managed in the Configuration Item List
 \item define the Product Baseline
 \item support changes to Configuration Items within the DMCCB
 \item manage the delivery of software products
 \item maintain the Configuration Item List
 \item manage the configuration control resources used by DM
 \item be aware of the relation between  the elements of the Product Baseline (in order for instance to be able to answer the question: ``What is the environment and which software is installed?'')
 \item check that the Product Assurance  and CMP procedures are correctly applied when Configuration Items are changed
 \item participate in CCB activities
\end{itemize}

The Configuration Manager is the secretary of the CCB and 
works with the support of the Scientific and Technical Leaders 
and participates in the CCB monitoring the development 
and change control process.



\subsubsection{Configuration Item List}
Configuration Item List (CIL) is the list of Items that are maintained under configuration control.
CUs and DPC need to report their configuration items in the CIL with an adequate level of details.
CIL is part of the development plan but may be written 
in a separate document to which the development plan refers to.

The configuration manager in charge has to identify the configuration items to include in the CIL, 
with the help of the technical leader and to maintain it when changes to the configuration items happen.


\subsubsection{Release management\label{sect:relMng}}
In DM usually each product will be released once per cycle. 
Additional releases may be done in case of bug fixing, urgent issues, or in case that the previous one is incomplete.
In case of longer cycles, intermediate major releases can be done.

Each release needs to be identified with:

\begin{itemize}
\item Configuration Item
\item Documentation:
\subitem User Manual: to be updated each major release
\subitem Requirements Specification: to be updated each major release
\subitem Test Specification: to be updated each major release
\subitem Release Note: new document each major release, updated for patch releases
\subitem Test Report: new document each major release, updated for patch releases
\item Latest Release in the master branch in github.
\end{itemize}
This information identifies a product baseline.

%Release preparation is in charge of the DU leader and is subject to CCB approval.
The product manager  is in charge of preparing the release.
After CCB approval, the release will be delivered to NCSA.

\subsubsection{Configuration Baseline \label{sect:BSLdef}}
A Configuration Baseline (CB) represents the approved status of the project at key milestones like formal review 
or at the beginning of test activities.

Configuration Baselines are applicable to hardware and software, and will 
include the documents that describe the CIs and their status.


\subsection{Lead Institution Senior Positions}
Each Lead Institution has a Project Manager and Scientific/Engineering Lead, who jointly have overall end product responsibility for a broad area of DM work, typically a Work Breakdown Structure (WBS) Level 2 element. They are supervisors of the team at that institution.  Their roles and responsibilities are similar to the DM Project Manager, DM Project Scientist, and DM System Architect, and DM QA and Test Lead, but within the scope of work assigned to that institution.  These leaders are required to acknowledge and implement direction from the DM leadership in all matters pertaining to the DM project.  The DM Project Manager and DM Project Scientist have direct input into the performance appraisals of the Institution Project Manager and Scientific/Engineering Lead. 
The lead institutions are covered  in \secref{sect:leadtutes}.

%\subsection{ \label{role:}}

\section{Teams within Data Management} \label{sect:groups}

Since the DM team is distributed in terms of geography and responsibility across the LSST partner and lead institutions, mechanisms are needed to ensure that the project remains on track at all times. There are five primary coordinating bodies to ensure the management, technical, and quality integrity of the DM Subsystem.

\subsection{System Science Team \label{sect:dmsst}}

Members of the DM System Science Team (SST) work together to define, maintain, and communicate to the DM Systems Engineering team a coherent vision of the LSST DM system responsive to the overall LSST Project goals, as well as scientifically validate the as-built system (\citeds{LDM-503}, Section~9.).

\begin{figure}[htbp]
\begin{center}
\includegraphics[width=0.6\textwidth]{images/DmSSTOrg}
\caption{DM System Science Team organisation.
\label{fig:sstorg}}
\end{center}
\end{figure}



\subsubsection{Organization and Goals}

The System Science Team includes:
\begin{itemize}
\item DM Subsystem Scientist (chair)
\item DM Science Validation Scientist
\item DM Institutional Science Leads
\item DM System Science Analysts
\item DM Science Pipelines Scientist
\end{itemize}

The System Science Team has been chartered to:
\begin{itemize}
\item Support the DM Subsystem Scientist (as the overall DM Product Owner) in ensuring that Data Management Subsystem's initiatives provide solutions that meet the overall LSST science goals.
\item Support the Institutional Science Leads in their roles as Product Owners for elements of the DM system their respective institutions have been tasked to deliver.
\item Support the DM Science Validation Scientist, who organizes and coordinates the science validation efforts (\citeds{LDM-503}).
\item Guide the work of System Science Analysts, who generally lead and/or execute studies needed to support SST work.
\item Provide a venue for communication with the Science Pipelines Scientist, who broadly advises on topics related to the impact of science pipelines on delivered science and vice versa (\secref{role:pipe}).
\end{itemize}

The members of the System Science Team report to the DM Subsystem Scientist and share the following responsibilities:
\begin{itemize}
\item Communicate with the science community and internal stakeholders to understand their needs, identifying the aspects to be satisfied by the DM Subsystem.
\item Liaise with the science collaborations to understand and coordinate any concurrent science investigations relevant to the DM Subsystem.
\item Develop, maintain, and articulate the vision of DM-delivered LSST data products and services that is responsive to stakeholder needs, balanced across science areas, well motivated, and scientifically and technologically current.
\item Work with the DM Project Manager and DM Technical Managers to communicate and articulate the DM System vision and requirements to the DM engineering team.
\item Identify, develop, and champion new scientific opportunities for the LSST DM System, as well as identify risks where possible.
\item Develop change proposals and/or evaluate the scientific impact of proposed changes to DM deliverables driven by schedule, budget, or other constraints.
\item Lead the Science Verification of the deliverables of the DM subsystem.
\end{itemize}

\subsubsection{Regression Monitoring of KPMs and other Metrics}

All KPMs and other regression monitoring metrics will be calculated on a regular cadence (daily if possible).
They are monitored by the SQuaRE scientist, with status periodically reported to the System Science Team (SST).
The SQuaRE scientist brings up any major regressions to the attention of the SST, along with an initial assessment of the problem.
The SST has the responsibility of monitoring the overall system for whether it meets its key performance metrics as well as understanding any significant performance regressions in performance.
The SST may recommend further actions to the DM Project Manager and/or Scientist, if necessary.
These include performing additional testing, broader root cause analysis, documenting the regression, or recommendations on the priority of fixing the regression relative to presently scheduled work.

\subsubsection{Communications}

DM System Science Team communication mechanisms are described on the SST Confluence page at \url{http://ls.st/sst}.

\subsection{DM Systems Engineering Team \label{sect:sysengt}}

The Systems Engineering Team is led by the DMPM (\secref{role:dmpm}) and looks after all aspects of systems engineering.
It is comprised of not only the Systems Engineer (\secref{role:sysengineer}), but also the Software Architect (\secref{role:softarc}), Operations Architect (\secref{role:opsarc}), DM Subsystem Scientist (\secref{role:dmps}), Pipeline Scientist (\secref{role:pipe}), Interface Scientist (\secref{role:dmis}), and the DM Deputy Project Manager (\secref{role:ddmpm}).

While the product owners (\secref{role:prodo}) help DM to create products which are fit for purpose, the Systems Engineering Team must ensure we do it correctly. This group concerns itself with (sub)system wide decisions on architecture and software engineering.

The specific tasks of this group include:

\begin{itemize}
\item Formalize the product list for DM\footnote{In this sense, ``products'' are the software and systems which produce data products, rather than the data products themselves. See also \ref{sect:products}.}
\item Formalize the documentation tree for DM, defining which documents need to be produced for each product
\item Agree the process for tracing the baseline requirements verification and validation status.
\item Agree the formal versions of documents and software which form the technical baseline, individual items will go through the CCB for formal approval.  This includes upload to docushare.
\end{itemize}

Some of these tasks are will be delegated to individual group members.
These individuals also are the conduit to/from the rest of the DM team to raise ideas/issues with the engineering approach.

\subsubsection{Communications}

The Systems Engineering Team will only physically meet to discuss specific topics: there will not be a regular meeting of the group outside of the one to one meetings with the DM project manager for the individuals in the group.
Discussions will be held via email until in person talks are required.

\subsection{DM Leadership Team \label{sect:dmlt}}

The purpose of the DM Leadership Team (DMLT) is to assist the DMPM  establish the scope of work and resource allocation across DM and ensure overall project management integrity across DM.
The following mandate established the DMLT:

\begin{itemize}
\item Charter/purpose
	\begin{itemize}
	\item Maintain scope of work and keep within resource allocation across DM
	\item Ensure overall project management integrity across DM
	\item Ensure Earned Value management requirements are met
	\end{itemize}
\item Membership
	\begin{itemize}
	\item Co-chaired by the DM Project Manager (\secref{role:dmpm}) and DM Project Scientist (\secref{role:dmps})
	\item Lead Institution Technical/Control Account Managers (T/CAMs; \secref{role:tcam})
	\item Institutional Science or Engineering Leads (\secref{role:scilead})
	\item Members of the DM Systems Engineering Team (\secref{sect:sysengt})
	\end{itemize}
\item Responsibilities
	\begin{itemize}
	\item Prepares all budgets, schedules, plans
	\item Meets every week to track progress, address issues/risks, adjust work assignments and schedules, and disseminate/discuss general PM communications
	\end{itemize}
\end{itemize}

The DM Leadership Team and the DM Systems Engineering Team (\secref{sect:sysengt}) work in synchrony.
The DMLT makes sure the requirements and architecture/design are estimated and scheduled in accordance with LSST Project required budgets and schedules.

 \subsubsection{Communications}
A mailing list\footnote{\url{lsst-dmlt@listserv.lsstcorp.org}} exists for DMLT related messages.
On Mondays the DMLT hold a brief (30 to 45 minutes) telecon. This serves to:

\begin{itemize}
\item Allow the Project manager and DM Scientist  to pass on important project level information and general guidance.
\item Raise any blocking or priority issues across DM --- this may result in calling a splinter meeting to further discuss with relevant parties.
\item Inform all team members of any change requests (LCRs) in process at LSST level which may be of interest to or have an impact on DM
\item Check on outstanding actions on DMLT members
\end{itemize}

Face to Face meetings of DM are held twice a year; these are opportunities to:

\begin{itemize}
\item Discuss detailed planning for the next cycle
\item Discuss technical topics in a face to face environment
\item Work together on critical issues
\item Help make DM function as a team
\end{itemize}

\subsection{DM Change Control Board \label{sect:dmccb}}

The DMCCB has responsibility for issues similar to those of the LSST Change Control Board, but with its scope restricted to the DM Subsystem.
The DMCCB reviews and approves changes to all baselines in the Subsystem, including proposed changes to the DM System Requirements (DMSR), reference design, sizing model, i.e. any LDM-series document.
The Technical Baseline, including software/hardware and documentation, is written by DM and controlled by the DMCCB.
DMCCB validates that the form and content of the Technical Baseline is consistent with LSST project standards such as the Systems Engineering Management Plan (SEMP) \citeds{LSE-17}.

\begin{itemize}
\item Charter/purpose:
	\begin{itemize}
	\item Ensure that the DM Technical Baseline (LDM-xxx) documents are baselined and subsequently changed only when necessary and according to LSST and DM configuration control processes
	\item Monitor and approve all DM software releases
	\end{itemize}
\item Membership:
	\begin{itemize}
        \item Core members:
                \begin{itemize}
                \item DM Project Manager
                \item DM Subsystem Scientist
	        \item Chaired by the Systems Engineer (\secref{role:sysengineer}).
                \item Operations Architect
                \item Software Architect
                \item DM Configuration and Release Manager (secretary)
                \end{itemize}
        \item Optional members are involved when topics to discuss are relevant to their areas of expertise:
                \begin{itemize}
                \item DMPM Deputy, when DMPM is not available
                \item DM Deputy Subsystem Scientist, when DM Subsystem Scientist is not available
                \item Pipeline Scientist
                \item Science Pipelines / Interface Scientist
                \item T/CAMs, who can delegate to their deputies
                \end{itemize}
	\item For on-line virtual meetings, if a consensus or quorum is not reached within one week, the DM Project Manager will make a unilateral decision
        \item DMPM can also make unilateral decisions in cases of urgency. In that case DMCCB will assess the change a \textit{posteriori}.
	\end{itemize}
\item Responsibilities:
	\begin{itemize}
	\item Determine when deliverables (controlled documents and software) are of sufficient maturity and quality to be baselined (placed under configuration controlled status) or released.
	\item Review and approve/reject proposed changes to baselined items
        \item Approve new or unplanned releases, approve the content of the releases, overview the release process.
        \item Review and approves all RFCs prior to 'Adoption'
        \item Monitor DM Jira issues status
	\end{itemize}
\end{itemize}

The DM-CCB will meet, physically or virtually, every week for 30 minutes. Agenda will be available beforehand.

\subsection{DM Science Validation Team}
\label{sect:dmsvt}

The DM Science Validation Team guides the definition of, and receives the products of, science validation and dress rehearsal activities, following the long-term roadmap described in \citeds{LDM-503}.
Decisions on the strategic goals of these activities are made in conjunction with the DM Subsystem Scientist and Project Manager.

The DM Science Validation Team is chaired by the DM Science Validation Scientist (\secref{role:dmsvs}).
Its membership includes the DM Pipelines Scientist (\secref{role:pipe}) and the various Institutional Science/Engineering Leads (\secref{role:scilead}).
Depending on the activities currently being executed, other members of the System Science Team (\secref{sect:dmsst}), the wider DM Construction Project, and/or external experts may be temporarily added to the team.

\section{Lead institutions in \gls{DM} \label{sect:leadtutes}}

\subsection{LSST Tucson\label{sect:tucson}}

The \gls{LSST} Project Office in Tucson hosts the \gls{DM} \gls{Project Manager} (\secref{role:dmpm}) and the \gls{Systems Engineer} (\secref{role:sysengineer}).
In addition, it is home to the Science Quality and Reliability Engineering (\gls{SQuaRE}) group and \gls{LSST} International Communications and Base Site (\gls{ICBS}) groups, described below.

\subsubsection{Science Quality and Reliability Engineering \label{sect:square}}

The \gls{SQuaRE} group is primarily charged with providing technical feedback to the \gls{DM} \gls{Project Manager} that demonstrates that \gls{DM} is fulfilling its responsibilities with regard to quality — of both scientific data products and software — software performance, and reliability. As such, areas of activity include:

\begin{itemize}

\item Development of algorithms to detect and analyze quality issues with data\footnote{This may overlap with work carried out by the \gls{Science Pipelines} groups (\S\S\ref{sect:ap} \& \ref{sect:drp}). In some instances this will involve sharing code; in others, it may merit duplicating a \gls{metric} to ensure that it is correct.}

\item Infrastructure development to support the generation, collection, and analysis of data quality and performance metrics

\item \gls{DM} developer support services to ensure \gls{DM} is using appropriate tools to aid software quality

\item \gls{DM} documentation support, to include defining standards and providing tooling for documentation as well as some document writing

\item Support of publicly released software products, including porting and distributing them according to the scientific community's needs

\end{itemize}

In the event that \gls{SQuaRE} identifies issues with the performance or future maintainability of the \gls{DM} codebase, it will bring them to the attention of the \gls{DM} Software Architect. In the event that \gls{SQuaRE} identifies issues with the quality of the data or algorithmic performance, it will bring them to the attention of the \gls{DM} \gls{Subsystem Scientist}.

\subsubsection{LSST International Communications and Base Site}
The \gls{ICBS} group spans both Tucson and La Serena, and is responsible for the design, procurement, installation, deployment, verification, and operating support during construction and commissioning of all data communications networks at the \gls{Summit} and Base sites, as well as links between all the \gls{LSST} Sites, with two exceptions:  the \gls{Summit} Network (\gls{WBS} 1.04C.12.5) and the \gls{Archive} External Network (1.02C.07.04.06).  In the case of the exceptions, there are technical and managerial interfaces between the \gls{ICBS} and the responsible parties, as well as overlaps of staff.  The \gls{LSST} Network Engineering Team (\gls{NET}) spans all of these networking assignees and is chaired by the \gls{ICBS} staff.

The \gls{ICBS} group is also jointly responsible with the Data Facility Team at \gls{NCSA} for procurement, installation, deployment, verification, and operating support during construction and commissioning of the computing and storage infrastructure at the Base Site.

Since a large majority of the \gls{ICBS} work involves procurement and contracted services, the group works in close cooperation with \gls{AURA} procurement and contracts, as well as with the following major sub-awardees and their subcontractors:

\begin{itemize}
	\item \gls{REUNA}: Chilean National Networks
	\item Florida International University/AmLight: International Networks connecting Chile and the United States, and \gls{US} National Networks.
\end{itemize}

\subsection {Princeton University \label{sect:princeton}}

Princeton University hosts the Pipelines Scientist (\secref{role:pipe}) and the Data \gls{Release} Production group, described below.

\subsubsection{Data \gls{Release} Production \label{sect:drp}}

The Data \gls{Release} Production (\gls{DRP}) group has three major areas of activity within \gls{DM}.

\begin{itemize}

  \item{Definition and implementation of the scientific algorithms and pipelines which will be used to generate \gls{LSST}'s annual data releases;}

  \item{Definition and implementation of the algorithms and pipelines which will be used to produce the ``calibration products'' (for example, flat fields, characterization of detector effects, etc) which will be used as inputs to the photometric \gls{calibration} procedure in both nightly and annual data processing. This includes the development of the spectrophotometric data reduction \gls{pipeline} for the Auxiliary Telescope;}

  \item{Development, in conjunction with the \gls{Alert Production} team (\gls{AP}; \secref{sect:ap}), of a library of re-usable software libraries and components which form the basis of both the \gls{AP} and \gls{DRP} pipelines and which are made available to science users within the \gls{LSST} \gls{Science Platform}.}

\end{itemize}

Development of software in support of annual data releases and of reusable software components are carried out under the direction of the \gls{DRP} Science Lead, who acts as product owner for this part of the system.
The \gls{DRP} Science Lead is ultimately responsible to both the Pipelines Scientist (\secref{role:pipe}) and \gls{DM} \gls{Subsystem Scientist} (\secref{role:dmps}).

The product owner for the \gls{calibration} products is the \gls{LSST} \gls{Calibration Scientist} (who doubles as the Pipelines Scientist, \secref{role:pipe}).
The \gls{Calibration Scientist} liaises with other \gls{LSST} subsystems and with the products owners of the annual and nightly data processing pipelines to ensure that appropriate \gls{calibration} products are available to those pipelines to enable them to meet specifications.

Management of the group is the responsibility of the Deputy \gls{Science Pipelines} \gls{T/CAM}, reporting to the \gls{Science Pipelines} \gls{T/CAM} and ultimately to the \gls{DM} \gls{Project Manager} (\secref{role:dmpm}).

The \gls{DRP} group is responsible for delivering software which adheres to the architectural and testing standard defined by the Software Architect (\secref{role:softarc}).
In addition, the \gls{DRP} group is responsible for testing each major product delivered to demonstrate its fitness for purpose, and working with the \gls{DM} \gls{Subsystem Scientist} and \gls{DM} System Science Team (\secref{sect:dmsst}) to define, run and analyze ``data challenges'' and other large scale tests to validate the performance of the data release production system.

\subsection {The University of Washington\label{sect:uw}}

\subsubsection{Alert Production\label{sect:ap}}

The \gls{Alert Production} (\gls{AP}) group has 4 major areas of activity within \gls{DM}.

\begin{itemize}

  \item{Definition and implementation of the scientific algorithms and pipelines which will be used to generate alerts from \gls{LSST}'s image stream.  This will serve as the alert generation \gls{pipeline};}

  \item{Definition and implementation a scalable and reliable system for transmitting the alerts generated by the alert generation \gls{pipeline} including a mechanism for applying simple filters to the stream. This is the alert distribution and filtering system;}

  \item{Definition and implementation of a system for identifying moving objects in our solar system and fitting their physical properties. This is the Moving Objects Processing System (\gls{MOPS});}

  \item{Development, in conjunction with the Data \gls{Release} Production team (\gls{DRP}; \secref{sect:drp}), of a library of re-usable software libraries and components which form the basis of both the \gls{AP} and \gls{DRP} pipelines and which are made available to science users within the \gls{LSST} \gls{Science Platform}.}

\end{itemize}

Development of software in support of the alert generation \gls{pipeline}, alert distribution system, \gls{MOPS} and of reusable software components are carried out under the direction of the \gls{AP} Science Lead, who acts as product owner for this part of the system.
The \gls{AP} Science Lead is ultimately responsible to both the Pipelines Scientist (\secref{role:pipe}) and \gls{DM} \gls{Subsystem Scientist} (\secref{role:dmps}).

Management of the group is the responsibility of the \gls{Science Pipelines} \gls{T/CAM}, reporting to the \gls{DM} \gls{Project Manager} (\secref{role:dmpm}).

The \gls{AP} group is responsible for delivering software which adheres to the architectural and testing standard defined by the Software Architect (\secref{role:softarc}).
In addition, the \gls{AP} group is responsible for testing each major product delivered to demonstrate its fitness for purpose, and working with the \gls{DM} \gls{Subsystem Scientist} and \gls{DM} System Science Team (\secref{sect:dmsst}) to define, run and analyze ``data challenges'' and other large scale tests to validate the performance of the data release production system.

\subsection {California Institute of Technology/IPAC\label{sect:ipac}}
IPAC hosts the \gls{LSST} \gls{Science Platform} Scientist (\secref{role:scip}), the \gls{DM} Interface Scientist (\secref{role:dmis}), and the Science User Interface and Tools (\gls{SUIT}) group described below.

\subsubsection{ Science User Interface and Tools}

The Science User Interface and Tools (\gls{SUIT}) group has four major areas of activity within \gls{DM}:

Design and develop the \gls{Firefly} Web-based visualization and data exploration framework, based upon the the same software already in operations in other \gls{NASA} archive services (i.e. \gls{IRSA}’s \gls{WISE} Image Service) . The \gls{Firefly} framework provides three basic components –  image display and manipulation, tabular table display and manipulation, and \gls{2D} plotting – all of which work together to provide different views into the same data. \gls{Firefly} also provides JavaScript and Python APIs to enable developers to easily use the components in their own Web pages or Jupyter notebooks.

Develop the interfaces needed to connect Firefly to the other LSST Science Platform components, e.g., connect to authentication and authorization, DAX services, user workspace, flexible compute system.  Develop visualizations of the objects in the LSST Data Products data model, and support their metadata; e.g., Footprint, HeavyFootprint, WCS models.  Provide basic access to Firefly from the LSST stack via afw.display.

Design and implement the Portal Aspect of the \gls{LSST} \gls{Science Platform} for \gls{Data Access Center}, based on \gls{Firefly}, providing scientists an easy to use interface to search, visualize, and explore \gls{LSST} data. The portal will enable users to do as much data discovery and exploration as possible through complex searches and facilitate data assessment through visualization and interaction.  The Portal will assist users in understanding the semantic linkages between the various \gls{LSST} data products. The Portal will guide users to documentation on the \gls{Science Platform} itself, the \gls{LSST} data products, and the processing that generated them.  Support linkage between the Portal and Notebook aspects of the \gls{Science Platform}, enabling users to switch between the aspects easily by providing tools to make data selected in the Portal readily available for further analysis in user notebooks.

Design and develop the \gls{LSST} \gls{Alert} Subscription web portal to enable scientists to access the alert system. The subscription service will enable users to register filters and destinations for alerts matching their interests. The \gls{Alert} portal will also provide basic capabilities for searching alerts history and for exploring linkage between alerts and other data products.




\subsection {SLAC\label{sect:slac}}
SLAC hosts the \gls{DM} Software Architect (\secref{role:softarc}) and the Science Data \gls{Archive} and Data Access
Services group described below.

\subsubsection{Science Data \gls{Archive} and Data Access Services \label{sect:dax}}

The Science Data \gls{Archive} and Data Access Services (\gls{DAX}) group has the following major areas of activity
within \gls{DM}:

\begin{itemize}

  \item{Provides software to support ingestion, indexing, query, and administration of \gls{DM} catalog and image
  data products, data \gls{provenance}, and other associated \gls{metadata} within the \gls{LSST} Data Access Centers;}

  \item{Provides implementations of data access services (including \gls{IVOA} services), as well as associated
  client libraries, to be hosted within the \gls{LSST} Data Access Centers, which facilitate interaction between
  \gls{LSST} data products and tools provided by both other parts of the \gls{LSST} project and by the astronomical
  research community at large;}

  \item{Provides a Python framework (the ``Data \gls{Butler}''), used by the \gls{LSST} science pipelines, to facilitate
  abstract persistence/retrieval of in-memory Python objects to/from generic archives of those objects;}

  \item{Provides a Python framework (``SuperTask'') which serves as an interface layer between \gls{pipeline}
  orchestration and algorithmic code, and which allows pipelines to be constructed, configured, and run at
  the level of a single node or a group of tightly-synchronized nodes;}

  \item{Provides support for various middleware and infrastructure toolkits used by \gls{DM} which would otherwise
  have no authoritative home institution within DM (e.g. logging support library, spherical geometry support
  library).}

\end{itemize}

Management of the group is the responsibility of the \gls{DAX} \gls{T/CAM}, reporting to the \gls{DM} \gls{Project Manager}
(\secref{role:dmpm}).

The \gls{DAX} group is responsible for delivering software which adheres to the architectural and testing standard
defined by the Software Architect (\secref{role:softarc}). In addition, the \gls{DAX} group is responsible for
testing each major product delivered to demonstrate its fitness for purpose, and running and analyzing large
scale tests to validate the performance of the science data archive and data access systems.

\subsection {NCSA\label{sect:ncsa}}


NCSA hosts the \gls{LSST} Project Office Information Security Officer and Computer Security group, as well as the \gls{DM} group responsible for construction and integration of the \gls{LSST} Data Facility (\gls{LDF}), described below.

The \gls{LDF} group has the following major areas of activity within \gls{DM}:

\begin{enumerate}
	\item	\gls{Construction} of services, including software and operational methods, supporting observatory operations and nightly data production (Level 1 Services). Level 1 Services ingest raw data from all Observatory cameras and the Engineering and Facilities Database (\gls{EFD}) into the central archive; provide a dedicated computing service controllable by the Observatory Control System (\gls{OCS}) for prompt generation of nightly \gls{calibration} assessments, science image parameters, and \gls{transient} alerts; and provide computing services, data access, and a \gls{QA} portal for Observatory staff.
	\item	\gls{Construction} of services, including software and operational methods, for bulk batch data production. \gls{Batch Production} Services execute processing campaigns, using resources at \gls{NCSA} and satellite computing centers, to produce data release products, generate templates and calibrations, and perform scaled testing of science pipelines to assess production readiness.
	\item	\gls{Construction} of services, including software and operational methods, for hosting and operating data access services for community users. These services host the \gls{SUIT} portal, manage the JupyterLab environment, provide computing and data storage for the Data Access Centers, enable bulk data export, and host the \gls{LSST} limited alert-filtering service and feeds to community-provided brokers.
	\item	\gls{Construction} of services, including software and operational methods, for the \gls{Data Backbone}. \gls{Data Backbone} Services provide ingestion, management, distribution, access, integrity checking, and backup and disaster recovery for files and catalog data in the \gls{LSST} central data archive.
	\item	\gls{Construction} and operation of services for \gls{LSST} staff. Staff Services provide specific testing and integration platforms (e.g., a Prototype \gls{Data Access Center}) and general computing and data services for \gls{LSST} developers.
	\item	Provisioning and management of hardware infrastructure at NCSA and the Chilean Base Center for all services described above, as well as infrastructure for project-wide network-based computer security services and authentication and authorization services.
	\item	\gls{Construction} and operation of a service management framework and methods to monitor operations of service elements in accordance with service level agreements, track issues, manage service availability, and support change management.
	\item	Operation of services and \gls{IT} systems during construction to support on-going development, integration, and commissioning activities.
\end{enumerate}
The \gls{LDF} group is responsible for delivering instantiated production services, which integrate software and hardware components developed across \gls{DM}. The \gls{LDF} group performs large-scale tests to integrate and verify production readiness of all components.

\section{Development Process} \label{sect:devproc}

DM is essentially a large software project; in particular we are developing scientific software with the uncertainties that brings with it.
An Agile process \citep{it:agile} is particularly suited to scientific software development.  The development follows a six month  cyclical approach and  DM  products are under continuous
integration using the Jenkins tool. All code is developed in the GitHub open source repository under an open source license.
Releases follow a six-month cadence but the code on the master branch is intended to be always working with the continuous integration system ensuring this.

How this fits with the Earned Value System is described in \citeds{DMTN-020}.


\subsection{Communications}

The main stories for the six-month planning period are discussed at the DMLT F2F meeting near the beginning of the cycle (See \secref{sect:dmlt}).

The T/CAMs of each of the institutions meet via video on Tuesdays and Fridays for a short \emph{standup} meeting to ensure that any cross-team issues are surfaced and resolved expeditiously.
This meeting is chaired by the Deputy Project Manager.
Each T/CAM notes any significant progress of interest to other teams and any problems or potential problems that may arise.

\subsection{Conventions}
Coding guidelines and conventions are documented online in \url{https://developer.lsst.io}

\subsection{Reviews} \label{sect:reviews}

The DM Project Manager and Subsystem Scientist will periodically convene internal reviews (following \citeds{LSE-159})
of major DM components as necessary to assess progress and maintain the integrity of the overall system. Planned DM reviews will be listed at the LSST Project Review Hub (\url{https://project.lsst.org/reviews/hub/}).
%\begin{itemize}

%\item  Science and Alerts Pipelines Review
%\item   Verification Plan Review
  %\item  Science platform, perhaps in 3 parts
	%\begin{itemize}
	  %\item  JupyterLab
	  %\item SUI portal
	  %\item Web/APIs
	%\end{itemize}
  %\item  Calibration Review
%\end{itemize}

In addition, smaller components of the system will undergo DM-internal design reviews.  The DMPM decides what will be reviewed (with input from all DM members) and is the Decision Making Authority for approving review recommendations.  Participants in the design review will normally include all members of the DMCCB and other experts as appropriate (e.g. the LSST Information Security Officer or designated substitute if there are any security implications).  The design review will check that the design:
\begin{itemize}
\item meets the requirements and satisfies the use cases, and an implementation can be verified as doing so
\item conforms to the LSST DM architecture and has well-defined interfaces
\item is expected to be efficient in terms of labor cost, non-labor cost, and schedule
\item is expected to be reliable, maintainable, supportable, usable, and secure
\item conforms to good engineering practices
\end{itemize}

Design review presentations should include:
\begin{itemize}
\item the identification of the components under review in terms of where they fit within the overall architecture
\item use cases and requirements applicable to the components under review that show how they will be used and how they respond/support all usage
\item an API or other description of the public interfaces to the components under review
\item a description of the internal patterns and algorithms to be used in the design, known limitations to those, and justification why the limitations are acceptable for this development
\item a description of the technological approach to implementation, including use of any third-party components, and reuse of existing elements (e.g. this will be a specialization of the XYZ framework classes)
\item a description of how the function and performance of the component(s) under review will be tested
\end{itemize}

\section{Data Management Problem/Conflict Resolution }
The above organizational structure allocates significant responsibility to lead institutions.
As such, when problems arise that cannot be solved with the responsibility and scope allocated to an institution, the path of escalation and resolution of such problems must be clear.

Any inter-institutional issues should be brought as early as possible to the DM Project Manager.
The PM will attempt to mediate a resolution.
The PM may consult with the DMLT, DM System Science Team and DM Systems Engineering Team if there are scientific or technical impacts to be considered.

Should an issue need to be escalated the PM will bring it up in the weekly LSST Project Managers Meeting.
In that forum a way forward will be agreed with the LSST Project Manager and other subsystem managers.


\appendix
\newpage
\section{DM Product List \label{sect:prodlist}}
%%%%%%%%%%%%%%%%%%%%%%%%%%%%%%%%%%%%%%%%%%%%%%%%%%%%%%%%%%%%%%%%%%%%%%%%%%%%%%%%%%%%%%%%%%%%%%%%%%%  Product table generated by makeProductTree.py do not modify.
%%%%%%%%%%%%%%%%%%%%%%%%%%%%%%%%%%%%%%%%%%%%%%%%%%%%%%%%%%%%%%%%%%%%%%%%%%%%%%%%%%%%%%%%%%%%%%%%%
\begin{longtable}{|p{0.08\textwidth}|p{0.18\textwidth}|p{0.4\textwidth}|p{0.14\textwidth}|p{0.14\textwidth}|p{0.14\textwidth}|}\hline 
 \bf WBS & Product & Description & Manager & Owner & Packages\\ \hline   
{\tiny .} & {\small Data Management} & Data Management System &  &  & \\ \hline 
{\tiny 1.02C.06.02.01} & {\small Data Butler Client} & Data Butler data access client library & Fritz Mueller &  & \\ \hline 
{\tiny .} & {\small Data Access Center} & DAC Software &  &  & \\ \hline 
{\tiny 1.02C.03.03} & {\small Alert DB} & Alert database & Simon Krughoff & Eric Bellm & \\ \hline 
{\tiny } & {\small Bulk Distrib} & Bulk Distribution System & Joel Plutchak &  & \\ \hline 
{\tiny } & {\small Proposal Manager} & Proposal Manager & Joel Plutchak &  & \\ \hline 
{\tiny } & {\small Resource Manager} & DAC Resource Manager & Joel Plutchak &  & \\ \hline 
{\tiny .} & {\small DAX Web Services} & DAX Web services including VO interfaces &  &  & \\ \hline 
{\tiny 1.02C.06.02.05} & {\small Catalog Access} & Catalog access & Fritz Mueller &  & \\ \hline 
{\tiny 1.02C.06.02.05} & {\small Cat Metadata Acc} & Catalog metadata access & Fritz Mueller &  & \\ \hline 
{\tiny 1.02C.06.02.05} & {\small Img Metadata Acc} & Image metadata access & Fritz Mueller &  & \\ \hline 
{\tiny 1.02C.06.02.04} & {\small Image Access} & Image access & Fritz Mueller &  & \\ \hline 
{\tiny 1.02C.06.02.02} & {\small Web Framework} & Web services framework & Fritz Mueller &  & \\ \hline 
{\tiny 1.02C.06.01.01} & {\small L1 Catalog DB} & L1 catalog database & Fritz Mueller &  & \\ \hline 
{\tiny 1.02C.06.01.01} & {\small L2 Catalog DB} & L2 catalog database & Fritz Mueller &  & \\ \hline 
{\tiny .} & {\small Developer services} & Developer services &  &  & \\ \hline 
{\tiny 1.02C.10.02.03.01} & {\small Build/Unit Test} & Build and unit test service & Frossie Economou &  & \\ \hline 
{\tiny 1.02C.10.02.03.04} & {\small Devel Comm Tools} & Developer communication tools & Frossie Economou &  & \\ \hline 
{\tiny 1.02C.10.02.03.03} & {\small Doc Infrastructure} & Documentation infrastructure & Frossie Economou &  & \\ \hline 
{\tiny 1.02C.10.02.03.01} & {\small SW Version Control} & Software version control system & Frossie Economou &  & \\ \hline 
{\tiny 1.02C.10.02.03.05} & {\small Issue Tracking} & Issue (ticket) tracking service & Frossie Economou &  & \\ \hline 
{\tiny 1.02C.10.02.03.02} & {\small Packaging/Distrib} & Packaging and distribution & Frossie Economou &  & \\ \hline 
{\tiny } & {\small Identity Manager} & Identity (Authentication and Authorization) Manager & Joel Plutchak &  & \\ \hline 
{\tiny .} & {\small Infrastructure} & Infrastructure Software Systems &  &  & \\ \hline 
{\tiny } & {\small Batch Proc} & Batch Processing System & Joel Plutchak &  & \\ \hline 
{\tiny .} & {\small Data Backbone} & Data Backbone System &  &  & \\ \hline 
{\tiny } & {\small Transport/Repl} & File and database transport and replication & Joel Plutchak &  & \\ \hline 
{\tiny 1.02C.06.02.05} & {\small Global Metadata} & Global metadata service & Fritz Mueller &  & \\ \hline 
{\tiny 1.02C.06.01.01} & {\small Provenance DB} & Provenance database & Fritz Mueller &  & \\ \hline 
{\tiny } & {\small Infra Systems} & Filesystems/ provisioning/monitoring systems and system management & Joel Plutchak &  & \\ \hline 
{\tiny 1.02C.06.02.03} & {\small Qserv DBMS} & Qserv distributed database system & Fritz Mueller &  & \\ \hline 
{\tiny } & {\small Integration Test} & Automated integration and test services &  &  & \\ \hline 
{\tiny .} & {\small IT Environments} & Computing and Storage Infrastructure including provisioning &  &  & \\ \hline 
{\tiny .} & {\small Archive IT} & Archive IT Environments &  &  & \\ \hline 
{\tiny } & {\small Archive Center Env} & Archive Production Center environment & Joel Plutchak &  & \\ \hline 
{\tiny } & {\small Archive DAC Env} & Archive DAC environment & Joel Plutchak &  & \\ \hline 
{\tiny } & {\small DAC Integ Env} & DAC Integration environment (PDAC) & Joel Plutchak &  & \\ \hline 
{\tiny } & {\small Archive DBB Env} & Archive Data Backbone endpoints and storage & Joel Plutchak &  & \\ \hline 
{\tiny } & {\small DBB Integ Env} & Data Backbone Integration environment & Joel Plutchak &  & \\ \hline 
{\tiny } & {\small Dev Env} & Developer environment & Joel Plutchak &  & \\ \hline 
{\tiny } & {\small L1 Integration Env} & Level 1 Integration environment & Joel Plutchak &  & \\ \hline 
{\tiny } & {\small L2 Integration Env} & Level 2 Integration environment & Joel Plutchak &  & \\ \hline 
{\tiny } & {\small Satellite Env} & Satellite compute environment & Joel Plutchak &  & \\ \hline 
{\tiny .} & {\small Base IT} & Base IT Environments &  &  & \\ \hline 
{\tiny } & {\small Base Center Env} & Base Production Center environment & Joel Plutchak &  & \\ \hline 
{\tiny } & {\small Base DAC Env} & Base DAC environment & Joel Plutchak &  & \\ \hline 
{\tiny } & {\small Base DBB Env} & Base Data Backbone endpoints and storage & Joel Plutchak &  & \\ \hline 
{\tiny .} & {\small Level 1 System} & Level 1 System &  &  & \\ \hline 
{\tiny 1.02C.03.03} & {\small Alert Broker Feed} & Alert broker feed service & Simon Krughoff & Eric Bellm & \\ \hline 
{\tiny } & {\small L1 Offline Proc} & L1 Offline Processing System & Joel Plutchak &  & \\ \hline 
{\tiny .} & {\small L1 OCS Components} & Level 1 Online (OCS-connected) &  &  & \\ \hline 
{\tiny } & {\small Archiver} & Archiving Commandable SAL Component & Joel Plutchak & Felipe Menanteau & \\ \hline 
{\tiny } & {\small Catchup Archiver} & Catch-up Archiving Commandable SAL Component & Joel Plutchak & Felipe Menanteau & \\ \hline 
{\tiny } & {\small EFD Tranform} & EFD Transformation Commandable SAL Component & Joel Plutchak & Felipe Menanteau & \\ \hline 
{\tiny } & {\small Header Generator} & Header Generator Commandable SAL Component & Joel Plutchak & Felipe Menanteau & \\ \hline 
{\tiny } & {\small OCS Batch Proc} & OCS-Driven Batch Processing Commandable SAL Component & Joel Plutchak & Felipe Menanteau & \\ \hline 
{\tiny } & {\small Pointing Publisher} & Pointing Prediction Publishing Commandable SAL Component & Joel Plutchak & Felipe Menanteau & \\ \hline 
{\tiny } & {\small Prompt Proc} & Prompt Processing Commandable SAL Component & Joel Plutchak & Felipe Menanteau & \\ \hline 
{\tiny } & {\small Telem Gateway} & Telemetry Gateway Commandable SAL Component & Joel Plutchak & Felipe Menanteau & \\ \hline 
{\tiny .} & {\small L1 Science Payloads} & L1 science payloads &  &  & \\ \hline 
{\tiny .} & {\small Offline Alert Prod} & Offline Alert Production payload &  &  & \\ \hline 
{\tiny 1.02C.03.03} & {\small Offline Alert Gen} & Offline alert generation pipeline & Simon Krughoff & Eric Bellm & \\ \hline 
{\tiny 1.02C.03.06} & {\small Moving Object} & Moving object pipeline & Simon Krughoff & Eric Bellm & \\ \hline 
{\tiny 1.02C.03.04} & {\small Precovery} & Precovery and forced photometry pipeline & Simon Krughoff & Eric Bellm & \\ \hline 
{\tiny 1.02C.03.01} & {\small Offline SFP} & Offline single frame processing pipeline & Simon Krughoff & Eric Bellm & \\ \hline 
{\tiny .} & {\small Prompt Alert Prod} & Prompt Processing Alert Production payload &  &  & \\ \hline 
{\tiny 1.02C.03.03} & {\small Alert Gen Pipe} & Alert generation pipeline & Simon Krughoff & Eric Bellm & \\ \hline 
{\tiny 1.02C.03.01} & {\small Single Frame Pipe} & Single frame processing pipeline & Simon Krughoff & Eric Bellm & \\ \hline 
{\tiny 1.02C.04.02} & {\small Aux Tel Spec Pipe} & Offline Auxiliary Telescope spectrograph pipeline & John Swinbank & Jim Bosch & \\ \hline 
{\tiny 1.02C.04.02} & {\small Daily Calibration} & OCS-Controlled batch daily CPP payload & John Swinbank & Jim Bosch & \\ \hline 
{\tiny 1.02C.04.02} & {\small Offline Calibration} & Offline calibration single frame processing pipeline & John Swinbank & Jim Bosch & \\ \hline 
{\tiny 1.02C.04.02} & {\small Prompt Calibration} & Prompt Processing calibration frame payload & John Swinbank & Jim Bosch & \\ \hline 
{\tiny 1.02C.04.02} & {\small CBP Control} & OCS control scripts for collimated beam projector control & John Swinbank & Jim Bosch & \\ \hline 
{\tiny } & {\small L1 Quality Control} & L1 QC measurement generators & Simon Krughoff & Eric Bellm & \\ \hline 
{\tiny .} & {\small Level 1 Services} & Level 1 Services &  &  & \\ \hline 
{\tiny } & {\small Aux Tel Archiver} & Auxiliary Telescope Archiving Service &  &  & \\ \hline 
{\tiny } & {\small ComCam Archiver} & ComCam Archiving Service &  &  & \\ \hline 
{\tiny } & {\small LSSTCam Archiver} & LSSTCam Archiving Service &  &  & \\ \hline 
{\tiny } & {\small ComCam Catchup} & ComCam Catchup Archiving Service &  &  & \\ \hline 
{\tiny } & {\small LSSTCam Catchup} & LSSTCam Catchup Archiving Service &  &  & \\ \hline 
{\tiny } & {\small ComCam Prompt} & ComCam Prompt Processing Service &  &  & \\ \hline 
{\tiny } & {\small LSSTCam Prompt} & LSSTCam Prompt Processing Service &  &  & \\ \hline 
{\tiny 1.02C.03.03} & {\small Alert Mini-Broker} & Alert mini-broker service & Simon Krughoff & Eric Bellm & \\ \hline 
{\tiny .} & {\small Level 2 System} & Level 2 System &  &  & \\ \hline 
{\tiny } & {\small L2 Quality Control} & L2 QC measurement generators & John Swinbank & Jim Bosch & \\ \hline 
{\tiny .} & {\small L2 Science Payloads} & L2 science payloads &  &  & \\ \hline 
{\tiny 1.02C.04.02} & {\small CPP Quality Control} & CPP QC measurement generators & John Swinbank & Jim Bosch & \\ \hline 
{\tiny 1.02C.04.02} & {\small Periodic Cal Prod} & Periodic CPP payload & John Swinbank & Jim Bosch & \\ \hline 
{\tiny 1.02C.04.02} & {\small Annual Cal Prod} & Annual CPP payload & John Swinbank & Jim Bosch & \\ \hline 
{\tiny .} & {\small Data Release Prod} & Annual mini-DRP and DRP payload &  &  & \\ \hline 
{\tiny 1.02C.04.04} & {\small Coadd and Diff} & Image coaddition and differencing & John Swinbank & Jim Bosch & \\ \hline 
{\tiny 1.02C.04.05} & {\small Coadd Processing} & Coadd processing & John Swinbank & Jim Bosch & \\ \hline 
{\tiny 1.02C.04.06} & {\small DRP Postprocessing} & DRP Postprocessing & John Swinbank & Jim Bosch & \\ \hline 
{\tiny 1.02C.04.03} & {\small Image Char and Cal} & Image characterization and calibration & John Swinbank & Jim Bosch & \\ \hline 
{\tiny 1.02C.04.06} & {\small Object Char} & Multi-epoch object characterization & John Swinbank & Jim Bosch & \\ \hline 
{\tiny 1.02C.04.05} & {\small Overlap Resolution} & Overlap resolution & John Swinbank & Jim Bosch & \\ \hline 
{\tiny 1.02C.06.01.01} & {\small DRP-Internal DB} & DRP-internal database & Fritz Mueller &  & \\ \hline 
{\tiny .} & {\small Production Exec} & Production Execution System &  &  & \\ \hline 
{\tiny } & {\small Campaign Manager} & Campaign Manager & Joel Plutchak &  & \\ \hline 
{\tiny } & {\small Job Activator} & Job Activator & Joel Plutchak &  & \\ \hline 
{\tiny } & {\small Pilot Activator} & Pilot Activator & Joel Plutchak &  & \\ \hline 
{\tiny } & {\small Workflow Manager} & Workflow Manager/Orchestrator & Joel Plutchak &  & \\ \hline 
{\tiny } & {\small Workload Manager} & Workload Manager & Joel Plutchak &  & \\ \hline 
{\tiny .} & {\small DM Networks} & Data Management Provided Networks &  &  & \\ \hline 
{\tiny 1.02C.07.04.06} & {\small Arc Extl Net} & Archive External Network & Don Petravick & D Wheeler & \\ \hline 
{\tiny 1.02C.07.04.03 (moving to 1.02C.08)} & {\small Base network} & Base Local Area Network  & Don Petravick (moving to Jeff Kantor) & Jeff Kantor/Don Petravick & \\ \hline 
{\tiny .} & {\small Chilean Nat} & Summit - Gatehouse La Serena - Gatehouse/ La Serena - Santiago Networks DWDM Equipment &  &  & \\ \hline 
{\tiny 1.02C.08.03.01.03} & {\small Summit Net} & Summit - AURA Gatehouse Network & Jeff Kantor & Jeff Kantor & \\ \hline 
{\tiny 1.02C.08.03.01.04} & {\small DWDM Equipment} & DWDM Equipment & Jeff Kantor & Jeff Kantor & \\ \hline 
{\tiny 1.02C.08.03.01.01A} & {\small La Serena Net} & La Serena - AURA Gatehouse Network & Jeff Kantor & Jeff Kantor & \\ \hline 
{\tiny 1.02C.08.03.01.01} & {\small La Ser - Santi } & La Serena - Santiago Network & Jeff Kantor & Jeff Kantor & \\ \hline 
{\tiny .} & {\small Int/US WANs} & International WAN/US WAN &  &  & \\ \hline 
{\tiny 1.02C.08.03.02.01} & {\small Miami 100 Gbps } & Santiago - Miami 100 Gbps Ring & Jeff Kantor & Jeff Kantor & \\ \hline 
{\tiny 1.02C.08.03.02.02} & {\small Network Mgmt} & Network Management & Jeff Kantor & Jeff Kantor & \\ \hline 
{\tiny 1.02C.08.03.02.03} & {\small Santiago - Boca } & Santiago - Boca Raton Spectrum & Jeff Kantor & Jeff Kantor & \\ \hline 
{\tiny 1.02C.08.03.02.01} & {\small US National WAN} & US National WAN & Jeff Kantor & Jeff Kantor & \\ \hline 
{\tiny 1.02C.08.03} & {\small Long-Haul Nets} & Summit - Base/ Base - Archive/ US Networks & Jeff Kantor & Jeff Kantor & \\ \hline 
{\tiny } & {\small Precursor Data} & Precursor data for development and testing &  &  & \\ \hline 
{\tiny .} & {\small Science Algorithms} & Common science algorithmic components &  &  & \\ \hline 
{\tiny 1.02C.04.05} & {\small Aperture Corr} & Aperture correction & John Swinbank & Jim Bosch & \\ \hline 
{\tiny 1.02C.03.01} & {\small Artifact Detection} & Artifact detection & Simon Krughoff & Eric Bellm & \\ \hline 
{\tiny 1.02C.03.01} & {\small Artifact Interp} & Artifact interpolation & Simon Krughoff & Eric Bellm & \\ \hline 
{\tiny 1.02C.04.05} & {\small Association/Match} & Association and matching & John Swinbank & Jim Bosch & \\ \hline 
{\tiny 1.02C.03.07} & {\small Astrometric Fit} & Astrometric fitting & Simon Krughoff & Eric Bellm & \\ \hline 
{\tiny 1.02C.03.06} & {\small Attribution/Precov} & Attribution and precovery & Simon Krughoff & Eric Bellm & \\ \hline 
{\tiny 1.02C.04.03} & {\small Background Estim} & Background estimation & John Swinbank & Jim Bosch & \\ \hline 
{\tiny 1.02C.04.03} & {\small Background Ref} & Background reference & John Swinbank & Jim Bosch & \\ \hline 
{\tiny 1.02C.03.02} & {\small DIAObj Association} & DIAObject association & Simon Krughoff & Eric Bellm & \\ \hline 
{\tiny 1.02C.03.04} & {\small DCR Template Gen} & DCR-corrected template generation & Simon Krughoff & Eric Bellm & \\ \hline 
{\tiny 1.02C.04.05} & {\small Deblending} & Deblending & John Swinbank & Jim Bosch & \\ \hline 
{\tiny } & {\small Img Decorrelation} & Image decorrelation & Simon Krughoff & Eric Bellm & \\ \hline 
{\tiny 1.02C.04.04} & {\small Image Coaddition} & Image coaddition & John Swinbank & Jim Bosch & \\ \hline 
{\tiny 1.02C.03.01} & {\small ISR} & ISR & Simon Krughoff & Eric Bellm & \\ \hline 
{\tiny 1.02C.04.05} & {\small Measurement} & Measurement & John Swinbank & Jim Bosch & \\ \hline 
{\tiny } & {\small Orbit/Ephemeris} & Orbit tools & Simon Krughoff & Eric Bellm & \\ \hline 
{\tiny 1.02C.03.06} & {\small Ephemeris Calc} & Ephemeris calculation & Simon Krughoff & Eric Bellm & \\ \hline 
{\tiny 1.02C.03.06} & {\small Orbit Fitting} & Orbit fitting & Simon Krughoff & Eric Bellm & \\ \hline 
{\tiny 1.02C.03.06} & {\small Orbit Merging} & Orbit merging & Simon Krughoff & Eric Bellm & \\ \hline 
{\tiny 1.02C.03.06} & {\small Tracklet Gen} & Tracklet generation & Simon Krughoff & Eric Bellm & \\ \hline 
{\tiny 1.02C.03.08} & {\small Photometric Fit} & Photometric fitting & Simon Krughoff & Eric Bellm & \\ \hline 
{\tiny } & {\small Proper Motion} & Proper motion and parallax & Simon Krughoff & Eric Bellm & \\ \hline 
{\tiny 1.02C.04.03} & {\small PSF Estim Large} & PSF estimation (visit) & John Swinbank & Jim Bosch & \\ \hline 
{\tiny 1.02C.03.01} & {\small PSF Estim Small} & PSF estimation (1 CCD) & Simon Krughoff & Eric Bellm & \\ \hline 
{\tiny 1.02C.04.04} & {\small PSF Matching} & PSF matching & John Swinbank & Jim Bosch & \\ \hline 
{\tiny } & {\small Raw Meas Cal} & Raw measurement calibration & John Swinbank & Jim Bosch & \\ \hline 
{\tiny 1.02C.03.01} & {\small Reference Catalogs} & Reference catalogs & Simon Krughoff & Eric Bellm & \\ \hline 
{\tiny 1.02C.03.02} & {\small Reference Match} & Matching to reference catalogs & Simon Krughoff & Eric Bellm & \\ \hline 
{\tiny 1.02C.03.01} & {\small Spatial Models} & Spatial models & Simon Krughoff & Eric Bellm & \\ \hline 
{\tiny 1.02C.04.05} & {\small Source Detection} & Source detection & John Swinbank & Jim Bosch & \\ \hline 
{\tiny 1.02C.04.05} & {\small Star/Galaxy Sep} & Star/galaxy classification & John Swinbank & Jim Bosch & \\ \hline 
{\tiny 1.02C.03.04} & {\small Template Storage} & Difference template storage/retrieval & Simon Krughoff & Eric Bellm & \\ \hline 
{\tiny } & {\small Variability Char} & Variability characterization & Simon Krughoff & Eric Bellm & \\ \hline 
{\tiny .} & {\small Science Platform} & Science Platform &  &  & \\ \hline 
{\tiny .} & {\small SciPlat Notebook} & Science Platform notebook component &  &  & \\ \hline 
{\tiny 1.02C.10.02.02.05} & {\small Notebook Activator} & Notebook Activators & Frossie Economou &  & \\ \hline 
{\tiny 1.02C.10.02.02.06} & {\small Notebook Deploy} & Notebook deployment & Frossie Economou &  & \\ \hline 
{\tiny 1.02C.10.02.02.01} & {\small Notebook Env} & Basic notebook environment & Frossie Economou &  & \\ \hline 
{\tiny 1.02C.05.07.04} & {\small Notebook SUIT Intf} & Notebook visualization widgets and other Notebook/Portal bridges  & Xiuqin Wu &  & \\ \hline 
{\tiny 1.02C.10.02.02.04} & {\small Notebook SW Env} & Notebook software environments & Frossie Economou &  & \\ \hline 
{\tiny .} & {\small SciPlat Portal} & Science Platform portal component &  &  & \\ \hline 
{\tiny 1.02C.05.07.03} & {\small Firefly Python APIs} & Low-level Python API to Firefly & Xiuqin Wu &  & \\ \hline 
{\tiny 1.02C.05.06 } & {\small Firefly} & LSST-independent Firefly framework and visualization capabilities & Xiuqin Wu &  & \\ \hline 
{\tiny 1.02C.05.09} & {\small SUI Alert Interfaces} & Portal alert interfaces to configure alert subscriptions & Xiuqin Wu &  & \\ \hline 
{\tiny 1.02C.05.08} & {\small Portal Applications} & Web application(s) implementing the Portal & Xiuqin Wu &  & \\ \hline 
{\tiny  user workspace} & {\small Portal Interfaces} & Interfaces to DAX &  identity management & 1.02C.05.07 & \\ \hline 
{\tiny 1.02C.05.07.03} & {\small Visualizers} & Firefly components to visualize LSST Science Pipelines data objects & Xiuqin Wu &  & \\ \hline 
{\tiny .} & {\small Science Primitives} & Science software primitives &  &  & \\ \hline 
{\tiny 1.02C.03.05} & {\small Camera Descr} & Camera descriptions & Simon Krughoff & Eric Bellm & \\ \hline 
{\tiny 1.02C.03.05} & {\small Chromaticity Utils} & Chromaticity utilities & Simon Krughoff & Eric Bellm & \\ \hline 
{\tiny 1.02C.04.01} & {\small Convolution} & Convolution kernels & John Swinbank & Jim Bosch & \\ \hline 
{\tiny 1.02C.03.05} & {\small Approx 2-D Fields} & Interpolation and approximation of 2-D fields & Simon Krughoff & Eric Bellm & \\ \hline 
{\tiny 1.02C.04.01} & {\small Footprints} & Footprints & John Swinbank & Jim Bosch & \\ \hline 
{\tiny 1.02C.03.05} & {\small Fourier Transforms} & Fourier transforms & Simon Krughoff & Eric Bellm & \\ \hline 
{\tiny 1.02C.03.05} & {\small Common Functions} & Common functions and source profiles & Simon Krughoff & Eric Bellm & \\ \hline 
{\tiny .} & {\small Geometry} & Geometry gathering &  &  & \\ \hline 
{\tiny 1.02C.03.05} & {\small Cartesian Geom} & Cartesian geometry & Simon Krughoff & Eric Bellm & \\ \hline 
{\tiny 1.02C.03.05} & {\small Coord Transforms} & Coordinate transformations & Simon Krughoff & Eric Bellm & \\ \hline 
{\tiny 1.02C.06.04.03} & {\small Spherical Geom} & Spherical geometry & Fritz Mueller &  & \\ \hline 
{\tiny 1.02C.04.01} & {\small Images} & Images & John Swinbank & Jim Bosch & \\ \hline 
{\tiny 1.02C.04.01} & {\small MC Sampling} & Monte Carlo sampling & John Swinbank & Jim Bosch & \\ \hline 
{\tiny 1.02C.04.01} & {\small Num Integration} & Numerical integration & John Swinbank & Jim Bosch & \\ \hline 
{\tiny 1.02C.04.01} & {\small Num Optimization} & Numerical optimization & John Swinbank & Jim Bosch & \\ \hline 
{\tiny 1.02C.04.01} & {\small PhotoCal Repr} & Photometric calibration representation & John Swinbank & Jim Bosch & \\ \hline 
{\tiny 1.02C.06.02.01} & {\small Property/Metadata} & Multi-type associative containers & Fritz Mueller &  & \\ \hline 
{\tiny 1.02C.03.05} & {\small Point-Spread Funcs} & Point-spread functions & Simon Krughoff & Eric Bellm & \\ \hline 
{\tiny 1.02C.04.01} & {\small Random Numbers} & Random number generation & John Swinbank & Jim Bosch & \\ \hline 
{\tiny 1.02C.04.01} & {\small Science Tools} & Science tools & John Swinbank & Jim Bosch & \\ \hline 
{\tiny 1.02C.04.01} & {\small Basic Statistics} & Basic statistics & John Swinbank & Jim Bosch & \\ \hline 
{\tiny 1.02C.04.01} & {\small Tables} & Tables & John Swinbank & Jim Bosch & \\ \hline 
{\tiny 1.02C.03.05} & {\small Tree Structures} & Tree structures (for searching) & Simon Krughoff & Eric Bellm & \\ \hline 
{\tiny 1.02C.04.01} & {\small Warping} & Warping & John Swinbank & Jim Bosch & \\ \hline 
{\tiny .} & {\small QC Dashboard} & QC measurement collection/storage/dashboard service &  &  & \\ \hline 
{\tiny 1.02C.10.02.01.04} & {\small Alert QC} & Alert stream QC harness & Frossie Economou &  & \\ \hline 
{\tiny 1.02C.10.02.01.01} & {\small QC Harness} & QC harness & Frossie Economou &  & \\ \hline 
{\tiny 1.02C.10.02.01.02} & {\small QC Notifications} & QC threshold notification framework & Frossie Economou &  & \\ \hline 
{\tiny 1.02C.10.02.01.03} & {\small QC Reports} & QC verification reporting & Frossie Economou &  & \\ \hline 
{\tiny .} & {\small Task Execution} & Task execution framework &  &  & \\ \hline 
{\tiny 1.02C.06.03} & {\small Activator Bases} & Activator base and Command Line Activator & Fritz Mueller &  & \\ \hline 
{\tiny 1.02C.06.03} & {\small Pipeline Config} & Pipeline configuration & Fritz Mueller &  & \\ \hline 
{\tiny 1.02C.06.04.01} & {\small Logging} & Logging & Fritz Mueller &  & \\ \hline 
{\tiny 1.02C.06.03} & {\small Multi-Core Task} & Multi-core Task API & Fritz Mueller &  & \\ \hline 
{\tiny 1.02C.06.03} & {\small Multi-Node Task} & Multi-node Task API & Fritz Mueller &  & \\ \hline 
{\tiny 1.02C.06.03} & {\small SuperTask} & SuperTask & Fritz Mueller &  & \\ \hline 
\end{longtable} 

\newpage
\input{wbslist}
\newpage
\section{DM Discussion and Decision Making Process}
\label{sect:ddmp}

DM has adopted a multi-layered approach to making decisions.
In general, decisions are made at the lowest level possible within the team --- at the level of the individual developer where practical.
When this is not possible, decision making is escalated through the
hierarchy described below.

DM team members are empowered to make decisions as outlined in \href{https://developer.lsst.io/team/empowerment.html}{the Developer Guide : Empowerment}

The \gls{RFC} process is detailed in \href{https://developer.lsst.io/communications/rfc.html}{The developer guide : RFC }.



\newpage
\section{Traceability matrix of DMSR requirements to components \label{sect:tracefor}}
\begin{longtable}{|p{0.5\textwidth}|p{0.5\textwidth}|}\hline
{\bf Requirement} & {\bf Components}\\ \hline
DMS-REQ-0059 Bad Pixel Map & Offline Processing, Data Release Production Execution, Daily Calibration Update, Data Release Production, Raw Calibration
DMS-REQ-0060 Bias Residual Image & Offline Processing, Data Release Production Execution, Daily Calibration Update, Data Release Production, Raw Calibration
DMS-REQ-0061 Crosstalk Correction Matrix & Offline Processing, Data Release Production Execution, Daily Calibration Update, Data Release Production, Raw Calibration
DMS-REQ-0062 Illumination Correction Frame & Offline Processing, Data Release Production Execution, Daily Calibration Update, Data Release Production, Raw Calibration
DMS-REQ-0063 Monochromatic Flatfield Data Cube & Offline Processing, Data Release Production Execution, Daily Calibration Update, Data Release Production, Raw Calibration
DMS-REQ-0130 Calibration Data Products & Offline Processing, Data Release Production Execution, Daily Calibration Update, Data Release Production, Raw Calibration
DMS-REQ-0132 Calibration Image Provenance & Offline Processing, Data Release Production Execution, Daily Calibration Update, Data Release Production, Raw Calibration
DMS-REQ-0282 Dark Current Correction Frame & Offline Processing, Data Release Production Execution, Daily Calibration Update, Data Release Production, Raw Calibration
DMS-REQ-0283 Fringe Correction Frame & Offline Processing, Data Release Production Execution, Daily Calibration Update, Data Release Production, Raw Calibration
DMS-REQ-0018 Raw Science Image Data Acquisition & OCS Driven Batch
DMS-REQ-0020 Wavefront Sensor Data Acquisition & OCS Driven Batch
DMS-REQ-0022 Crosstalk Corrected Science Image Data Acquisition & Telemetry Gateway
DMS-REQ-0024 Raw Image Assembly & OCS Driven Batch
DMS-REQ-0068 Raw Science Image Metadata & OCS Driven Batch
DMS-REQ-0265 Guider Calibration Data Acquisition & OCS Driven Batch, Offline Processing, Data Release Production Execution, Daily Calibration Update, Data Release Production, Raw Calibration
DMS-REQ-0326 Storing Approximations of Per-pixel Metadata & MOPS and Forced Photometry
DMS-REQ-0331 Computing Derived Quantities & LSP JupyterLab, MOPS and Forced Photometry
DMS-REQ-0332 Denormalizing Database Tables & LSP JupyterLab
DMS-REQ-0333 Maximum Likelihood Values and Covariances & Task Execution Framework, Annual Calibration, MOPS and Forced Photometry
DMS-REQ-0346 Data Availability & OCS Driven Batch, Offline Processing, Prompt Processing, Telemetry Gateway, Data Release Production Execution, Data Access Center
DMS-REQ-0347 Measurements in catalogs & Task Execution Framework, LSP JupyterLab, Annual Calibration, MOPS and Forced Photometry
DMS-REQ-0004 Nightly Data Accessible Within 24 hrs & Alert Filtering, Image and EFD Archiving, Prompt Processing, Telemetry Gateway, Annual Calibration, Periodic Calibration
DMS-REQ-0010 Difference Exposures & Annual Calibration
DMS-REQ-0074 Difference Exposure Attributes & Annual Calibration
DMS-REQ-0069 Processed Visit Images & Annual Calibration
DMS-REQ-0029 Generate Photometric Zeropoint for Visit Image & Annual Calibration
DMS-REQ-0030 Generate WCS for Visit Images & Annual Calibration
DMS-REQ-0070 Generate PSF for Visit Images & Annual Calibration
DMS-REQ-0072 Processed Visit Image Content & Annual Calibration
DMS-REQ-0327 Background Model Calculation & Annual Calibration
DMS-REQ-0328 Documenting Image Characterization & Annual Calibration
DMS-REQ-0097 Level 1 Data Quality Report Definition & Science Algorithms, Annual Calibration
DMS-REQ-0099 Level 1 Performance Report Definition & Telemetry Gateway, Science Algorithms
DMS-REQ-0101 Level 1 Calibration Report Definition & Offline Processing, Data Release Production, Template Generation
DMS-REQ-0266 Exposure Catalog & Annual Calibration
DMS-REQ-0269 DIASource Catalog & Annual Calibration
DMS-REQ-0270 Faint DIASource Measurements & Annual Calibration
DMS-REQ-0271 DIAObject Catalog & Annual Calibration
DMS-REQ-0272 DIAObject Attributes & Annual Calibration
DMS-REQ-0273 SSObject Catalog & Periodic Calibration
DMS-REQ-0274 Alert Content & Annual Calibration
DMS-REQ-0317 DIAForcedSource Catalog & Annual Calibration, Periodic Calibration
DMS-REQ-0319 Characterizing Variability & Annual Calibration, Periodic Calibration
DMS-REQ-0323 Calculating SSObject Parameters & LSP JupyterLab
DMS-REQ-0324 Matching DIASources to Objects & LSP JupyterLab, Annual Calibration
DMS-REQ-0325 Regenerating L1 Data Products During Data Release Processing & MOPS and Forced Photometry
DMS-REQ-0034 Associate Sources to Objects & MOPS and Forced Photometry
DMS-REQ-0047 Provide PSF for Coadded Images & MOPS and Forced Photometry
DMS-REQ-0103 Produce Images for EPO & MOPS and Forced Photometry
DMS-REQ-0106 Coadded Image Provenance & MOPS and Forced Photometry
DMS-REQ-0267 Source Catalog & MOPS and Forced Photometry
DMS-REQ-0268 Forced-Source Catalog & MOPS and Forced Photometry
DMS-REQ-0275 Object Catalog & MOPS and Forced Photometry
DMS-REQ-0046 Provide Photometric Redshifts of Galaxies & MOPS and Forced Photometry
DMS-REQ-0276 Object Characterization & MOPS and Forced Photometry
DMS-REQ-0277 Coadd Source Catalog & MOPS and Forced Photometry
DMS-REQ-0349 Detecting extended  low surface brightness objects & MOPS and Forced Photometry
DMS-REQ-0278 Coadd Image Method Constraints & MOPS and Forced Photometry
DMS-REQ-0279 Deep Detection Coadds & MOPS and Forced Photometry
DMS-REQ-0280 Template Coadds & MOPS and Forced Photometry, Science Primitives
DMS-REQ-0281 Multi-band Coadds & MOPS and Forced Photometry
DMS-REQ-0329 All-Sky Visualization of Data Releases & MOPS and Forced Photometry
DMS-REQ-0330 Best Seeing Coadds & MOPS and Forced Photometry
DMS-REQ-0334 Persisting Data Products & Identity Manager, LSP JupyterLab
DMS-REQ-0335 PSF-Matched Coadds & MOPS and Forced Photometry
DMS-REQ-0336 Regenerating Data Products from Previous Data Releases & Identity Manager, LSP JupyterLab
DMS-REQ-0337 Detecting faint variable objects & MOPS and Forced Photometry
DMS-REQ-0338 Targeted Coadds & Identity Manager, LSP JupyterLab
DMS-REQ-0339 Tracking Characterization Changes Between Data Releases & Identity Manager, LSP JupyterLab
DMS-REQ-0320 Processing of Data From Special Programs & Science Platform, MOPS and Forced Photometry
DMS-REQ-0321 Level 1 Processing of Special Programs Data & Prompt Processing, Telemetry Gateway, Annual Calibration, Periodic Calibration
DMS-REQ-0322 Special Programs Database & Identity Manager, LSP JupyterLab
DMS-REQ-0344 Constraints on Level 1 Special Program Products Generation & Prompt Processing, Telemetry Gateway, Identity Manager, LSP JupyterLab, Annual Calibration, Periodic Calibration
DMS-REQ-0185 Archive Center & Proposal Manager, Identity Manager, Networks
DMS-REQ-0186 Archive Center Disaster Recovery & Proposal Manager, Identity Manager, Networks
DMS-REQ-0187 Archive Center Co-Location with Existing Facility & Networks
DMS-REQ-0188 Archive to Data Access Center Network & Workload+Workflow
DMS-REQ-0189 Archive to Data Access Center Network Availability & Workload+Workflow
DMS-REQ-0190 Archive to Data Access Center Network Reliability & Workload+Workflow
DMS-REQ-0191 Archive to Data Access Center Network Secondary Link & Workload+Workflow
DMS-REQ-0176 Base Facility Infrastructure & Identity Manager, Networks
DMS-REQ-0177 Base Facility Temporary Storage & Identity Manager
DMS-REQ-0178 Base Facility Co-Location with Existing Facility & Networks
DMS-REQ-0316 Commissioning Cluster & Networks
DMS-REQ-0180 Base to Archive Network & Workload+Workflow
DMS-REQ-0181 Base to Archive Network Availability & Workload+Workflow
DMS-REQ-0182 Base to Archive Network Reliability & Workload+Workflow
DMS-REQ-0183 Base to Archive Network Secondary Link & Workload+Workflow
DMS-REQ-0008 Pipeline Availability & Alert Filtering, Image and EFD Archiving, OCS Driven Batch, Offline Processing, Prompt Processing, Telemetry Gateway, Level 2 System, Data Release Production Execution, Data Access Center, Identity Manager, Networks, Workload+Workflow
DMS-REQ-0161 Optimization of Cost, Reliability and Availability in Order & Alert Filtering, Image and EFD Archiving, Identity Manager, Networks, Workload+Workflow, LSP JupyterLab, LSP Portal, QC System
DMS-REQ-0162 Pipeline Throughput & Alert Filtering, Image and EFD Archiving, OCS Driven Batch, Offline Processing, Prompt Processing, Telemetry Gateway, Identity Manager, Networks, Workload+Workflow
DMS-REQ-0163 Re-processing Capacity & Data Release Production Execution, Data Access Center, Identity Manager, Networks, Workload+Workflow
DMS-REQ-0164 Temporary Storage for Communications Links & Identity Manager
DMS-REQ-0165 Infrastructure Sizing for "catching up" & OCS Driven Batch, Prompt Processing, Identity Manager, Networks, Workload+Workflow
DMS-REQ-0166 Incorporate Fault-Tolerance & Identity Manager
DMS-REQ-0167 Incorporate Autonomics & Alert Filtering, Image and EFD Archiving, OCS Driven Batch, Offline Processing, Prompt Processing, Telemetry Gateway, Data Release Production Execution, Data Access Center, Identity Manager, Networks, Workload+Workflow
DMS-REQ-0314 Compute Platform Heterogeneity & Alert Filtering, Image and EFD Archiving, OCS Driven Batch, Offline Processing, Prompt Processing, Telemetry Gateway, Level 2 System, Data Release Production Execution, Data Access Center, Proposal Manager, Analysis and Developer Services, Integration and Test, Infrastructure, Identity Manager, ITC Environments, Networks, Workload+Workflow, Task Execution Framework, Science Platform, LSP JupyterLab, LSP Portal, QC System, Science Algorithms
DMS-REQ-0318 Data Management Unscheduled Downtime & Alert Filtering, Image and EFD Archiving, OCS Driven Batch, Offline Processing, Prompt Processing, Telemetry Gateway, Level 2 System, Data Release Production Execution, Data Access Center, Proposal Manager, Analysis and Developer Services, Identity Manager, ITC Environments, Networks, Workload+Workflow, LSP JupyterLab, LSP Portal, QC System, Science Algorithms
DMS-REQ-0122 Access to catalogs for external Level 3 processing & Proposal Manager, Identity Manager
DMS-REQ-0123 Access to input catalogs for DAC-based Level 3 processing & Identity Manager, LSP JupyterLab, LSP Portal, QC System
DMS-REQ-0124 Federation with external catalogs & Identity Manager, LSP JupyterLab, LSP Portal, QC System
DMS-REQ-0126 Access to images for external Level 3 processing & Proposal Manager, Identity Manager
DMS-REQ-0127 Access to input images for DAC-based Level 3 processing & Identity Manager, LSP JupyterLab, LSP Portal, QC System
DMS-REQ-0193 Data Access Centers & Proposal Manager, Identity Manager, Networks, Workload+Workflow, LSP JupyterLab, LSP Portal, QC System
DMS-REQ-0194 Data Access Center Simultaneous Connections & Workload+Workflow, LSP Portal, QC System
DMS-REQ-0196 Data Access Center Geographical Distribution & Networks
DMS-REQ-0197 No Limit on Data Access Centers & Identity Manager, ITC Environments, Networks, Workload+Workflow, LSP JupyterLab, LSP Portal, QC System
DMS-REQ-0075 Catalog Queries & LSP JupyterLab
DMS-REQ-0077 Maintain Archive Publicly Accessible & Identity Manager
DMS-REQ-0078 Catalog Export Formats & LSP JupyterLab
DMS-REQ-0094 Keep Historical Alert Archive & Identity Manager
DMS-REQ-0102 Provide Engineering & Facility Database Archive & OCS Driven Batch, Identity Manager
DMS-REQ-0309 Raw Data Archiving Reliability & OCS Driven Batch, Identity Manager
DMS-REQ-0310 Un-Archived Data Product Cache & Identity Manager
DMS-REQ-0311 Regenerate Un-archived Data Products & Identity Manager, LSP JupyterLab
DMS-REQ-0312 Level 1 Data Product Access & Telemetry Gateway, Identity Manager, LSP JupyterLab, Annual Calibration
DMS-REQ-0313 Level 1 & 2 Catalog Access & Identity Manager, LSP JupyterLab
DMS-REQ-0341 Providing a Precovery Service & Prompt Processing, QC System, Periodic Calibration
DMS-REQ-0345 Logging of catalog queries & LSP JupyterLab
DMS-REQ-0168 Summit Facility Data Communications & Workload+Workflow
DMS-REQ-0170 Prefer Computing and Storage Down & Networks
DMS-REQ-0315 DMS Communication with OCS & OCS Driven Batch, Offline Processing, Telemetry Gateway, Level 2 System
DMS-REQ-0171 Summit to Base Network & Workload+Workflow
DMS-REQ-0172 Summit to Base Network Availability & Workload+Workflow
DMS-REQ-0173 Summit to Base Network Reliability & Workload+Workflow
DMS-REQ-0174 Summit to Base Network Secondary Link & Workload+Workflow
DMS-REQ-0175 Summit to Base Network Ownership and Operation & Workload+Workflow
DMS-REQ-0002 Transient Alert Distribution & Alert Filtering, Image and EFD Archiving, Telemetry Gateway, Annual Calibration
DMS-REQ-0089 Solar System Objects Available Within Specified Time & Prompt Processing, Identity Manager, LSP JupyterLab, Periodic Calibration
DMS-REQ-0096 Generate Data Quality Report Within Specified Time & Science Algorithms
DMS-REQ-0098 Generate DMS Performance Report Within Specified Time & Science Algorithms
DMS-REQ-0100 Generate Calibration Report Within Specified Time & Science Algorithms, Data Release Production
DMS-REQ-0131 Calibration Images Available Within Specified Time & Offline Processing, Telemetry Gateway, Data Release Production, Template Generation
DMS-REQ-0284 Level-1 Production Completeness & OCS Driven Batch, Prompt Processing, Telemetry Gateway
DMS-REQ-0285 Level 1 Source Association & Annual Calibration
DMS-REQ-0286 SSObject Precovery & Periodic Calibration
DMS-REQ-0287 DIASource Precovery & Identity Manager, LSP JupyterLab, Periodic Calibration
DMS-REQ-0288 Use of External Orbit Catalogs & Annual Calibration, Periodic Calibration
DMS-REQ-0342 Alert Filtering Service & Image and EFD Archiving
DMS-REQ-0343 Performance Requirements for LSST Alert Filtering Service & Alert Filtering, Image and EFD Archiving
DMS-REQ-0348 Pre-defined alert filters & Image and EFD Archiving
DMS-REQ-0289 Calibration Production Processing & Offline Processing, Data Release Production Execution, Daily Calibration Update, Data Release Production, Raw Calibration
DMS-REQ-0006 Timely Publication of Level 2 Data Releases & 
DMS-REQ-0350 Associating Objects across data releases & Identity Manager, MOPS and Forced Photometry
DMS-REQ-0291 Query Repeatibility & Identity Manager, LSP JupyterLab
DMS-REQ-0292 Uniqueness of IDs Across Data Releases & Identity Manager, LSP JupyterLab
DMS-REQ-0293 Selection of Datasets & Identity Manager, LSP JupyterLab
DMS-REQ-0294 Processing of Datasets & Offline Processing, Prompt Processing, Telemetry Gateway, Data Release Production Execution, Data Access Center, Identity Manager
DMS-REQ-0295 Transparent Data Access & LSP JupyterLab
DMS-REQ-0119 DAC resource allocation for Level 3 processing & Proposal Manager, Analysis and Developer Services, LSP JupyterLab, LSP Portal, QC System
DMS-REQ-0120 Level 3 Data Product Self Consistency & Identity Manager, Data Access Client, LSP JupyterLab
DMS-REQ-0121 Provenance for Level 3 processing at DACs & Identity Manager, Data Access Client, Task Execution Framework, Science Platform, LSP JupyterLab
DMS-REQ-0125 Software framework for Level 3 catalog processing & Identity Manager, Data Access Client, Task Execution Framework, Science Platform, LSP JupyterLab
DMS-REQ-0128 Software framework for Level 3 image processing & Identity Manager, Data Access Client, Task Execution Framework, Science Platform, LSP JupyterLab
DMS-REQ-0290 Level 3 Data Import & Identity Manager, LSP JupyterLab
DMS-REQ-0340 Access Controls of Level 3 Data Products & Identity Manager, ITC Environments
DMS-REQ-0009 Simulated Data & Annual Calibration, Daily Calibration Update, Data Release Production, MOPS and Forced Photometry, Periodic Calibration, Raw Calibration, Template Generation, Science Primitives
DMS-REQ-0032 Image Differencing & Science Payloads, Annual Calibration, MOPS and Forced Photometry
DMS-REQ-0033 Provide Source Detection Software & Science Payloads, Annual Calibration, MOPS and Forced Photometry
DMS-REQ-0042 Provide Astrometric Model & Science Payloads, Annual Calibration, MOPS and Forced Photometry
DMS-REQ-0043 Provide Calibrated Photometry & Science Payloads, Annual Calibration, MOPS and Forced Photometry
DMS-REQ-0052 Enable a Range of Shape Measurement Approaches & Science Payloads, Annual Calibration, MOPS and Forced Photometry
DMS-REQ-0160 Provide User Interface Services & QC System
DMS-REQ-0296 Pre-cursor, and Real Data & Identity Manager, Task Execution Framework, Science Payloads
DMS-REQ-0351 Provide Beam Projector Coordinate Calculation Software & 
DMS-REQ-0308 Software Architecture to Enable Community Re-Use & Science Payloads, Annual Calibration, Daily Calibration Update, Data Release Production, MOPS and Forced Photometry, Periodic Calibration, Raw Calibration, Template Generation, Science Primitives
DMS-REQ-0065 Provide Image Access Services & LSP JupyterLab
DMS-REQ-0155 Provide Data Access Services & LSP JupyterLab
DMS-REQ-0298 Data Product and Raw Data Access & Identity Manager, LSP JupyterLab
DMS-REQ-0299 Data Product Ingest & Identity Manager, LSP JupyterLab
DMS-REQ-0300 Bulk Download Service & Proposal Manager, Identity Manager
DMS-REQ-0156 Provide Pipeline Execution Services & Data Access Client
DMS-REQ-0302 Production Orchestration & Data Access Client
DMS-REQ-0303 Production Monitoring & Data Access Client
DMS-REQ-0304 Production Fault Tolerance & Data Access Client, Science Platform
DMS-REQ-0158 Provide Pipeline Construction Services & Science Platform
DMS-REQ-0305 Task Specification & Science Platform
DMS-REQ-0306 Task Configuration & Science Platform
DMS-REQ-0297 DMS Initialization Component & Networks
DMS-REQ-0301 Control of Level-1 Production & Telemetry Gateway
DMS-REQ-0307 Unique Processing Coverage & Data Access Client

\end{longtable}
\section{Traceability matrix of components to DMSR requirements \label{sect:traceback}}
\begin{longtable}{|p{0.3\textwidth}|p{0.7\textwidth}|}\hline
{\bf Components} & {\bf Requirement}\\ \hline
Data Management & \\\hline
Level 1 System & \\\hline
Alert Distribution & DMS-REQ-0010 Difference Exposures, DMS-REQ-0161 Optimization of Cost, Reliability and Availability in Order, DMS-REQ-0162 Pipeline Throughput, DMS-REQ-0163 Re-processing Capacity, DMS-REQ-0314 Compute Platform Heterogeneity, DMS-REQ-0318 Data Management Unscheduled Downtime, DMS-REQ-0122 Access to catalogs for external Level 3 processing, DMS-REQ-0089 Solar System Objects Available Within Specified Time, DMS-REQ-0348 Pre-defined alert filters\\\hline
Alert Filtering & DMS-REQ-0010 Difference Exposures, DMS-REQ-0161 Optimization of Cost, Reliability and Availability in Order, DMS-REQ-0162 Pipeline Throughput, DMS-REQ-0163 Re-processing Capacity, DMS-REQ-0314 Compute Platform Heterogeneity, DMS-REQ-0318 Data Management Unscheduled Downtime, DMS-REQ-0122 Access to catalogs for external Level 3 processing, DMS-REQ-0089 Solar System Objects Available Within Specified Time, DMS-REQ-0343 Performance Requirements for LSST Alert Filtering Service, DMS-REQ-0348 Pre-defined alert filters, DMS-REQ-0289 Calibration Production Processing\\\hline
Image and EFD Archiving & DMS-REQ-0020 Wavefront Sensor Data Acquisition, DMS-REQ-0022 Crosstalk Corrected Science Image Data Acquisition, DMS-REQ-0068 Raw Science Image Metadata, DMS-REQ-0265 Guider Calibration Data Acquisition, DMS-REQ-0326 Storing Approximations of Per-pixel Metadata, DMS-REQ-0347 Measurements in catalogs, DMS-REQ-0161 Optimization of Cost, Reliability and Availability in Order, DMS-REQ-0163 Re-processing Capacity, DMS-REQ-0166 Incorporate Fault-Tolerance, DMS-REQ-0314 Compute Platform Heterogeneity, DMS-REQ-0318 Data Management Unscheduled Downtime, DMS-REQ-0122 Access to catalogs for external Level 3 processing, DMS-REQ-0309 Raw Data Archiving Reliability, DMS-REQ-0310 Un-Archived Data Product Cache, DMS-REQ-0171 Summit to Base Network, DMS-REQ-0285 Level 1 Source Association\\\hline
OCS Driven Batch & DMS-REQ-0060 Bias Residual Image, DMS-REQ-0061 Crosstalk Correction Matrix, DMS-REQ-0062 Illumination Correction Frame, DMS-REQ-0063 Monochromatic Flatfield Data Cube, DMS-REQ-0130 Calibration Data Products, DMS-REQ-0132 Calibration Image Provenance, DMS-REQ-0282 Dark Current Correction Frame, DMS-REQ-0283 Fringe Correction Frame, DMS-REQ-0018 Raw Science Image Data Acquisition, DMS-REQ-0326 Storing Approximations of Per-pixel Metadata, DMS-REQ-0347 Measurements in catalogs, DMS-REQ-0266 Exposure Catalog, DMS-REQ-0161 Optimization of Cost, Reliability and Availability in Order, DMS-REQ-0163 Re-processing Capacity, DMS-REQ-0314 Compute Platform Heterogeneity, DMS-REQ-0318 Data Management Unscheduled Downtime, DMS-REQ-0122 Access to catalogs for external Level 3 processing, DMS-REQ-0171 Summit to Base Network, DMS-REQ-0284 Level-1 Production Completeness, DMS-REQ-0006 Timely Publication of Level 2 Data Releases, DMS-REQ-0295 Transparent Data Access\\\hline
Offline Processing & DMS-REQ-0347 Measurements in catalogs, DMS-REQ-0010 Difference Exposures, DMS-REQ-0322 Special Programs Database, DMS-REQ-0185 Archive Center, DMS-REQ-0161 Optimization of Cost, Reliability and Availability in Order, DMS-REQ-0163 Re-processing Capacity, DMS-REQ-0166 Incorporate Fault-Tolerance, DMS-REQ-0314 Compute Platform Heterogeneity, DMS-REQ-0318 Data Management Unscheduled Downtime, DMS-REQ-0122 Access to catalogs for external Level 3 processing, DMS-REQ-0345 Logging of catalog queries, DMS-REQ-0096 Generate Data Quality Report Within Specified Time, DMS-REQ-0285 Level 1 Source Association, DMS-REQ-0295 Transparent Data Access\\\hline
Prompt Processing & DMS-REQ-0024 Raw Image Assembly, DMS-REQ-0347 Measurements in catalogs, DMS-REQ-0010 Difference Exposures, DMS-REQ-0101 Level 1 Calibration Report Definition, DMS-REQ-0322 Special Programs Database, DMS-REQ-0185 Archive Center, DMS-REQ-0161 Optimization of Cost, Reliability and Availability in Order, DMS-REQ-0163 Re-processing Capacity, DMS-REQ-0314 Compute Platform Heterogeneity, DMS-REQ-0318 Data Management Unscheduled Downtime, DMS-REQ-0122 Access to catalogs for external Level 3 processing, DMS-REQ-0313 Level 1 and 2 Catalog Access, DMS-REQ-0171 Summit to Base Network, DMS-REQ-0089 Solar System Objects Available Within Specified Time, DMS-REQ-0284 Level-1 Production Completeness, DMS-REQ-0285 Level 1 Source Association, DMS-REQ-0295 Transparent Data Access, DMS-REQ-0307 Unique Processing Coverage\\\hline
Telemetry Gateway & DMS-REQ-0161 Optimization of Cost, Reliability and Availability in Order, DMS-REQ-0318 Data Management Unscheduled Downtime, DMS-REQ-0122 Access to catalogs for external Level 3 processing, DMS-REQ-0171 Summit to Base Network\\\hline
Level 2 System & \\\hline
Calibration Products Production Execution & DMS-REQ-0060 Bias Residual Image, DMS-REQ-0061 Crosstalk Correction Matrix, DMS-REQ-0062 Illumination Correction Frame, DMS-REQ-0063 Monochromatic Flatfield Data Cube, DMS-REQ-0130 Calibration Data Products, DMS-REQ-0132 Calibration Image Provenance, DMS-REQ-0282 Dark Current Correction Frame, DMS-REQ-0283 Fringe Correction Frame, DMS-REQ-0018 Raw Science Image Data Acquisition, DMS-REQ-0326 Storing Approximations of Per-pixel Metadata, DMS-REQ-0347 Measurements in catalogs, DMS-REQ-0161 Optimization of Cost, Reliability and Availability in Order, DMS-REQ-0164 Temporary Storage for Communications Links, DMS-REQ-0314 Compute Platform Heterogeneity, DMS-REQ-0318 Data Management Unscheduled Downtime, DMS-REQ-0122 Access to catalogs for external Level 3 processing, DMS-REQ-0006 Timely Publication of Level 2 Data Releases, DMS-REQ-0295 Transparent Data Access\\\hline
Data Release Production Execution & DMS-REQ-0347 Measurements in catalogs, DMS-REQ-0161 Optimization of Cost, Reliability and Availability in Order, DMS-REQ-0164 Temporary Storage for Communications Links, DMS-REQ-0314 Compute Platform Heterogeneity, DMS-REQ-0318 Data Management Unscheduled Downtime, DMS-REQ-0122 Access to catalogs for external Level 3 processing, DMS-REQ-0295 Transparent Data Access\\\hline
Data Access Center & \\\hline
Bulk Distribution & DMS-REQ-0186 Archive Center Disaster Recovery, DMS-REQ-0187 Archive Center Co-Location with Existing Facility, DMS-REQ-0318 Data Management Unscheduled Downtime, DMS-REQ-0122 Access to catalogs for external Level 3 processing, DMS-REQ-0123 Access to input catalogs for DAC-based Level 3 processing, DMS-REQ-0127 Access to input images for DAC-based Level 3 processing, DMS-REQ-0194 Data Access Center Simultaneous Connections, DMS-REQ-0120 Level 3 Data Product Self Consistency, DMS-REQ-0156 Provide Pipeline Execution Services\\\hline
Proposal Manager & DMS-REQ-0318 Data Management Unscheduled Downtime, DMS-REQ-0122 Access to catalogs for external Level 3 processing, DMS-REQ-0120 Level 3 Data Product Self Consistency\\\hline
Analysis and Developer Services & \\\hline
Developer Services & DMS-REQ-0318 Data Management Unscheduled Downtime\\\hline
Integration and Test & DMS-REQ-0318 Data Management Unscheduled Downtime\\\hline
Infrastructure & \\\hline
Data Backbone & DMS-REQ-0335 PSF-Matched Coadds, DMS-REQ-0337 Detecting faint variable objects, DMS-REQ-0339 Tracking Characterization Changes Between Data Releases, DMS-REQ-0320 Processing of Data From Special Programs, DMS-REQ-0344 Constraints on Level 1 Special Program Products Generation, DMS-REQ-0185 Archive Center, DMS-REQ-0186 Archive Center Disaster Recovery, DMS-REQ-0187 Archive Center Co-Location with Existing Facility, DMS-REQ-0177 Base Facility Temporary Storage, DMS-REQ-0178 Base Facility Co-Location with Existing Facility, DMS-REQ-0161 Optimization of Cost, Reliability and Availability in Order, DMS-REQ-0162 Pipeline Throughput, DMS-REQ-0163 Re-processing Capacity, DMS-REQ-0164 Temporary Storage for Communications Links, DMS-REQ-0165 Infrastructure Sizing for "catching up", DMS-REQ-0166 Incorporate Fault-Tolerance, DMS-REQ-0167 Incorporate Autonomics, DMS-REQ-0314 Compute Platform Heterogeneity, DMS-REQ-0318 Data Management Unscheduled Downtime, DMS-REQ-0122 Access to catalogs for external Level 3 processing, DMS-REQ-0123 Access to input catalogs for DAC-based Level 3 processing, DMS-REQ-0124 Federation with external catalogs, DMS-REQ-0126 Access to images for external Level 3 processing, DMS-REQ-0127 Access to input images for DAC-based Level 3 processing, DMS-REQ-0193 Data Access Centers, DMS-REQ-0194 Data Access Center Simultaneous Connections, DMS-REQ-0075 Catalog Queries, DMS-REQ-0078 Catalog Export Formats, DMS-REQ-0102 Provide Engineering and Facility Database Archive, DMS-REQ-0309 Raw Data Archiving Reliability, DMS-REQ-0310 Un-Archived Data Product Cache, DMS-REQ-0311 Regenerate Un-archived Data Products, DMS-REQ-0312 Level 1 Data Product Access, DMS-REQ-0313 Level 1 and 2 Catalog Access, DMS-REQ-0341 Providing a Precovery Service, DMS-REQ-0096 Generate Data Quality Report Within Specified Time, DMS-REQ-0288 Use of External Orbit Catalogs, DMS-REQ-0291 Query Repeatibility, DMS-REQ-0292 Uniqueness of IDs Across Data Releases, DMS-REQ-0293 Selection of Datasets, DMS-REQ-0294 Processing of Datasets, DMS-REQ-0295 Transparent Data Access, DMS-REQ-0121 Provenance for Level 3 processing at DACs, DMS-REQ-0125 Software framework for Level 3 catalog processing, DMS-REQ-0128 Software framework for Level 3 image processing, DMS-REQ-0290 Level 3 Data Import, DMS-REQ-0340 Access Controls of Level 3 Data Products, DMS-REQ-0009 Simulated Data, DMS-REQ-0351 Provide Beam Projector Coordinate Calculation Software, DMS-REQ-0299 Data Product Ingest, DMS-REQ-0300 Bulk Download Service, DMS-REQ-0156 Provide Pipeline Execution Services\\\hline
Identity Manager & DMS-REQ-0318 Data Management Unscheduled Downtime, DMS-REQ-0122 Access to catalogs for external Level 3 processing, DMS-REQ-0075 Catalog Queries, DMS-REQ-0009 Simulated Data\\\hline
ITC Environments & DMS-REQ-0186 Archive Center Disaster Recovery, DMS-REQ-0187 Archive Center Co-Location with Existing Facility, DMS-REQ-0188 Archive to Data Access Center Network, DMS-REQ-0177 Base Facility Temporary Storage, DMS-REQ-0316 Commissioning Cluster, DMS-REQ-0180 Base to Archive Network, DMS-REQ-0161 Optimization of Cost, Reliability and Availability in Order, DMS-REQ-0162 Pipeline Throughput, DMS-REQ-0163 Re-processing Capacity, DMS-REQ-0164 Temporary Storage for Communications Links, DMS-REQ-0166 Incorporate Fault-Tolerance, DMS-REQ-0314 Compute Platform Heterogeneity, DMS-REQ-0318 Data Management Unscheduled Downtime, DMS-REQ-0122 Access to catalogs for external Level 3 processing, DMS-REQ-0194 Data Access Center Simultaneous Connections, DMS-REQ-0197 No Limit on Data Access Centers, DMS-REQ-0075 Catalog Queries, DMS-REQ-0315 DMS Communication with OCS, DMS-REQ-0301 Control of Level-1 Production\\\hline
Networks & DMS-REQ-0189 Archive to Data Access Center Network Availability, DMS-REQ-0190 Archive to Data Access Center Network Reliability, DMS-REQ-0191 Archive to Data Access Center Network Secondary Link, DMS-REQ-0176 Base Facility Infrastructure, DMS-REQ-0181 Base to Archive Network Availability, DMS-REQ-0182 Base to Archive Network Reliability, DMS-REQ-0183 Base to Archive Network Secondary Link, DMS-REQ-0008 Pipeline Availability, DMS-REQ-0161 Optimization of Cost, Reliability and Availability in Order, DMS-REQ-0162 Pipeline Throughput, DMS-REQ-0163 Re-processing Capacity, DMS-REQ-0164 Temporary Storage for Communications Links, DMS-REQ-0166 Incorporate Fault-Tolerance, DMS-REQ-0314 Compute Platform Heterogeneity, DMS-REQ-0318 Data Management Unscheduled Downtime, DMS-REQ-0122 Access to catalogs for external Level 3 processing, DMS-REQ-0194 Data Access Center Simultaneous Connections, DMS-REQ-0196 Data Access Center Geographical Distribution, DMS-REQ-0075 Catalog Queries, DMS-REQ-0170 Prefer Computing and Storage Down, DMS-REQ-0172 Summit to Base Network Availability, DMS-REQ-0173 Summit to Base Network Reliability, DMS-REQ-0174 Summit to Base Network Secondary Link, DMS-REQ-0175 Summit to Base Network Ownership and Operation, DMS-REQ-0002 Transient Alert Distribution\\\hline
Workload+Workflow & DMS-REQ-0121 Provenance for Level 3 processing at DACs, DMS-REQ-0125 Software framework for Level 3 catalog processing, DMS-REQ-0128 Software framework for Level 3 image processing, DMS-REQ-0290 Level 3 Data Import, DMS-REQ-0302 Production Orchestration, DMS-REQ-0303 Production Monitoring, DMS-REQ-0304 Production Fault Tolerance, DMS-REQ-0158 Provide Pipeline Construction Services\\\hline
Data Access Client & DMS-REQ-0346 Data Availability, DMS-REQ-0004 Nightly Data Accessible Within 24 hrs, DMS-REQ-0318 Data Management Unscheduled Downtime, DMS-REQ-0125 Software framework for Level 3 catalog processing, DMS-REQ-0128 Software framework for Level 3 image processing, DMS-REQ-0290 Level 3 Data Import, DMS-REQ-0351 Provide Beam Projector Coordinate Calculation Software\\\hline
Task Execution Framework & DMS-REQ-0321 Level 1 Processing of Special Programs Data, DMS-REQ-0318 Data Management Unscheduled Downtime, DMS-REQ-0125 Software framework for Level 3 catalog processing, DMS-REQ-0128 Software framework for Level 3 image processing, DMS-REQ-0290 Level 3 Data Import, DMS-REQ-0158 Provide Pipeline Construction Services, DMS-REQ-0305 Task Specification, DMS-REQ-0306 Task Configuration, DMS-REQ-0297 DMS Initialization Component\\\hline
Science Platform & \\\hline
DAX VO+ Services & DMS-REQ-0332 Denormalizing Database Tables, DMS-REQ-0333 Maximum Likelihood Values and Covariances, DMS-REQ-0004 Nightly Data Accessible Within 24 hrs, DMS-REQ-0324 Matching DIASources to Objects, DMS-REQ-0325 Regenerating L1 Data Products During Data Release Processing, DMS-REQ-0335 PSF-Matched Coadds, DMS-REQ-0337 Detecting faint variable objects, DMS-REQ-0339 Tracking Characterization Changes Between Data Releases, DMS-REQ-0320 Processing of Data From Special Programs, DMS-REQ-0344 Constraints on Level 1 Special Program Products Generation, DMS-REQ-0185 Archive Center, DMS-REQ-0162 Pipeline Throughput, DMS-REQ-0318 Data Management Unscheduled Downtime, DMS-REQ-0122 Access to catalogs for external Level 3 processing, DMS-REQ-0124 Federation with external catalogs, DMS-REQ-0126 Access to images for external Level 3 processing, DMS-REQ-0193 Data Access Centers, DMS-REQ-0194 Data Access Center Simultaneous Connections, DMS-REQ-0075 Catalog Queries, DMS-REQ-0077 Maintain Archive Publicly Accessible, DMS-REQ-0094 Keep Historical Alert Archive, DMS-REQ-0312 Level 1 Data Product Access, DMS-REQ-0313 Level 1 and 2 Catalog Access, DMS-REQ-0341 Providing a Precovery Service, DMS-REQ-0168 Summit Facility Data Communications, DMS-REQ-0096 Generate Data Quality Report Within Specified Time, DMS-REQ-0288 Use of External Orbit Catalogs, DMS-REQ-0292 Uniqueness of IDs Across Data Releases, DMS-REQ-0293 Selection of Datasets, DMS-REQ-0294 Processing of Datasets, DMS-REQ-0119 DAC resource allocation for Level 3 processing, DMS-REQ-0120 Level 3 Data Product Self Consistency, DMS-REQ-0121 Provenance for Level 3 processing at DACs, DMS-REQ-0125 Software framework for Level 3 catalog processing, DMS-REQ-0128 Software framework for Level 3 image processing, DMS-REQ-0290 Level 3 Data Import, DMS-REQ-0340 Access Controls of Level 3 Data Products, DMS-REQ-0155 Provide Data Access Services, DMS-REQ-0298 Data Product and Raw Data Access, DMS-REQ-0299 Data Product Ingest, DMS-REQ-0300 Bulk Download Service\\\hline
LSP JupyterLab & DMS-REQ-0162 Pipeline Throughput, DMS-REQ-0318 Data Management Unscheduled Downtime, DMS-REQ-0122 Access to catalogs for external Level 3 processing, DMS-REQ-0124 Federation with external catalogs, DMS-REQ-0126 Access to images for external Level 3 processing, DMS-REQ-0193 Data Access Centers, DMS-REQ-0194 Data Access Center Simultaneous Connections, DMS-REQ-0196 Data Access Center Geographical Distribution, DMS-REQ-0075 Catalog Queries, DMS-REQ-0120 Level 3 Data Product Self Consistency\\\hline
LSP Portal & DMS-REQ-0162 Pipeline Throughput, DMS-REQ-0318 Data Management Unscheduled Downtime, DMS-REQ-0122 Access to catalogs for external Level 3 processing, DMS-REQ-0124 Federation with external catalogs, DMS-REQ-0126 Access to images for external Level 3 processing, DMS-REQ-0193 Data Access Centers, DMS-REQ-0194 Data Access Center Simultaneous Connections, DMS-REQ-0196 Data Access Center Geographical Distribution, DMS-REQ-0075 Catalog Queries, DMS-REQ-0345 Logging of catalog queries, DMS-REQ-0120 Level 3 Data Product Self Consistency, DMS-REQ-0296 Pre-cursor, and Real Data\\\hline
QC System & DMS-REQ-0099 Level 1 Performance Report Definition, DMS-REQ-0101 Level 1 Calibration Report Definition, DMS-REQ-0318 Data Management Unscheduled Downtime, DMS-REQ-0122 Access to catalogs for external Level 3 processing, DMS-REQ-0098 Generate DMS Performance Report Within Specified Time, DMS-REQ-0100 Generate Calibration Report Within Specified Time, DMS-REQ-0131 Calibration Images Available Within Specified Time\\\hline
Science Algorithms & DMS-REQ-0033 Provide Source Detection Software, DMS-REQ-0042 Provide Astrometric Model, DMS-REQ-0043 Provide Calibrated Photometry, DMS-REQ-0052 Enable a Range of Shape Measurement Approaches, DMS-REQ-0160 Provide User Interface Services, DMS-REQ-0351 Provide Beam Projector Coordinate Calculation Software, DMS-REQ-0065 Provide Image Access Services\\\hline
Science Payloads & \\\hline
Alert Production & DMS-REQ-0346 Data Availability, DMS-REQ-0004 Nightly Data Accessible Within 24 hrs, DMS-REQ-0010 Difference Exposures, DMS-REQ-0074 Difference Exposure Attributes, DMS-REQ-0069 Processed Visit Images, DMS-REQ-0029 Generate Photometric Zeropoint for Visit Image, DMS-REQ-0030 Generate WCS for Visit Images, DMS-REQ-0070 Generate PSF for Visit Images, DMS-REQ-0072 Processed Visit Image Content, DMS-REQ-0327 Background Model Calculation, DMS-REQ-0328 Documenting Image Characterization, DMS-REQ-0097 Level 1 Data Quality Report Definition, DMS-REQ-0099 Level 1 Performance Report Definition, DMS-REQ-0269 DIASource Catalog, DMS-REQ-0270 Faint DIASource Measurements, DMS-REQ-0271 DIAObject Catalog, DMS-REQ-0272 DIAObject Attributes, DMS-REQ-0273 SSObject Catalog, DMS-REQ-0317 DIAForcedSource Catalog, DMS-REQ-0319 Characterizing Variability, DMS-REQ-0323 Calculating SSObject Parameters, DMS-REQ-0325 Regenerating L1 Data Products During Data Release Processing, DMS-REQ-0322 Special Programs Database, DMS-REQ-0185 Archive Center, DMS-REQ-0313 Level 1 and 2 Catalog Access, DMS-REQ-0089 Solar System Objects Available Within Specified Time, DMS-REQ-0286 SSObject Precovery, DMS-REQ-0342 Alert Filtering Service, DMS-REQ-0032 Image Differencing, DMS-REQ-0033 Provide Source Detection Software, DMS-REQ-0042 Provide Astrometric Model, DMS-REQ-0043 Provide Calibrated Photometry, DMS-REQ-0052 Enable a Range of Shape Measurement Approaches, DMS-REQ-0160 Provide User Interface Services, DMS-REQ-0065 Provide Image Access Services\\\hline
Annual Calibration & DMS-REQ-0060 Bias Residual Image, DMS-REQ-0061 Crosstalk Correction Matrix, DMS-REQ-0062 Illumination Correction Frame, DMS-REQ-0063 Monochromatic Flatfield Data Cube, DMS-REQ-0130 Calibration Data Products, DMS-REQ-0132 Calibration Image Provenance, DMS-REQ-0282 Dark Current Correction Frame, DMS-REQ-0283 Fringe Correction Frame, DMS-REQ-0018 Raw Science Image Data Acquisition, DMS-REQ-0326 Storing Approximations of Per-pixel Metadata, DMS-REQ-0006 Timely Publication of Level 2 Data Releases, DMS-REQ-0032 Image Differencing, DMS-REQ-0065 Provide Image Access Services\\\hline
Daily Calibration Update & DMS-REQ-0060 Bias Residual Image, DMS-REQ-0061 Crosstalk Correction Matrix, DMS-REQ-0062 Illumination Correction Frame, DMS-REQ-0063 Monochromatic Flatfield Data Cube, DMS-REQ-0130 Calibration Data Products, DMS-REQ-0132 Calibration Image Provenance, DMS-REQ-0282 Dark Current Correction Frame, DMS-REQ-0283 Fringe Correction Frame, DMS-REQ-0018 Raw Science Image Data Acquisition, DMS-REQ-0326 Storing Approximations of Per-pixel Metadata, DMS-REQ-0266 Exposure Catalog, DMS-REQ-0131 Calibration Images Available Within Specified Time, DMS-REQ-0284 Level-1 Production Completeness, DMS-REQ-0006 Timely Publication of Level 2 Data Releases, DMS-REQ-0032 Image Differencing, DMS-REQ-0065 Provide Image Access Services\\\hline
Data Release Production & DMS-REQ-0331 Computing Derived Quantities, DMS-REQ-0332 Denormalizing Database Tables, DMS-REQ-0346 Data Availability, DMS-REQ-0004 Nightly Data Accessible Within 24 hrs, DMS-REQ-0034 Associate Sources to Objects, DMS-REQ-0047 Provide PSF for Coadded Images, DMS-REQ-0103 Produce Images for EPO, DMS-REQ-0106 Coadded Image Provenance, DMS-REQ-0267 Source Catalog, DMS-REQ-0268 Forced-Source Catalog, DMS-REQ-0275 Object Catalog, DMS-REQ-0046 Provide Photometric Redshifts of Galaxies, DMS-REQ-0276 Object Characterization, DMS-REQ-0277 Coadd Source Catalog, DMS-REQ-0349 Detecting extended  low surface brightness objects, DMS-REQ-0278 Coadd Image Method Constraints, DMS-REQ-0279 Deep Detection Coadds, DMS-REQ-0280 Template Coadds, DMS-REQ-0281 Multi-band Coadds, DMS-REQ-0329 All-Sky Visualization of Data Releases, DMS-REQ-0330 Best Seeing Coadds, DMS-REQ-0334 Persisting Data Products, DMS-REQ-0336 Regenerating Data Products from Previous Data Releases, DMS-REQ-0338 Targeted Coadds, DMS-REQ-0321 Level 1 Processing of Special Programs Data, DMS-REQ-0291 Query Repeatibility, DMS-REQ-0032 Image Differencing, DMS-REQ-0033 Provide Source Detection Software, DMS-REQ-0042 Provide Astrometric Model, DMS-REQ-0043 Provide Calibrated Photometry, DMS-REQ-0052 Enable a Range of Shape Measurement Approaches, DMS-REQ-0160 Provide User Interface Services, DMS-REQ-0065 Provide Image Access Services\\\hline
MOPS and Forced Photometry & DMS-REQ-0010 Difference Exposures, DMS-REQ-0274 Alert Content, DMS-REQ-0319 Characterizing Variability, DMS-REQ-0323 Calculating SSObject Parameters, DMS-REQ-0322 Special Programs Database, DMS-REQ-0185 Archive Center, DMS-REQ-0345 Logging of catalog queries, DMS-REQ-0096 Generate Data Quality Report Within Specified Time, DMS-REQ-0287 DIASource Precovery, DMS-REQ-0288 Use of External Orbit Catalogs, DMS-REQ-0342 Alert Filtering Service, DMS-REQ-0032 Image Differencing, DMS-REQ-0065 Provide Image Access Services\\\hline
Periodic Calibration & DMS-REQ-0060 Bias Residual Image, DMS-REQ-0061 Crosstalk Correction Matrix, DMS-REQ-0062 Illumination Correction Frame, DMS-REQ-0063 Monochromatic Flatfield Data Cube, DMS-REQ-0130 Calibration Data Products, DMS-REQ-0132 Calibration Image Provenance, DMS-REQ-0282 Dark Current Correction Frame, DMS-REQ-0283 Fringe Correction Frame, DMS-REQ-0018 Raw Science Image Data Acquisition, DMS-REQ-0326 Storing Approximations of Per-pixel Metadata, DMS-REQ-0006 Timely Publication of Level 2 Data Releases, DMS-REQ-0032 Image Differencing, DMS-REQ-0065 Provide Image Access Services\\\hline
Raw Calibration & DMS-REQ-0266 Exposure Catalog, DMS-REQ-0284 Level-1 Production Completeness, DMS-REQ-0032 Image Differencing, DMS-REQ-0065 Provide Image Access Services\\\hline
Template Generation & DMS-REQ-0281 Multi-band Coadds, DMS-REQ-0032 Image Differencing, DMS-REQ-0065 Provide Image Access Services\\\hline
Science Primitives & DMS-REQ-0043 Provide Calibrated Photometry, DMS-REQ-0052 Enable a Range of Shape Measurement Approaches, DMS-REQ-0160 Provide User Interface Services, DMS-REQ-0308 Software Architecture to Enable Community Re-Use\\\hline

\end{longtable}

%\input{precon}
\newpage

\section{References\label{sect:references}}
\renewcommand{\refname}{}
\bibliography{local,lsst,gaia_livelink_valid,refs,books,refs_ads}

%\section{Acronyms}
%The following table has been generated from the on-line Gaia acronym list:
\newline\newline%decrement table counter so table sin doc start at 1.
\addtocounter{table}{-1}
\begin{longtable}{|l|p{0.8\textwidth}|}\hline 
\textbf{Acronym} & \textbf{Description}  \\\hline
AP&Alerts Production \\\hline
CB&Configuration Baseline \\\hline
CCB&Change Control Board \\\hline
CI&Configuration Item \\\hline
CIL&Configuration Item List \\\hline
CM&Configuration Management \\\hline
CMP&Configuration Management Plan \\\hline
CU&Coordination Unit (in DPAC) \\\hline
DAC&Data Access Centre \\\hline
DAX&Data access services \\\hline
DDMPM&Data Management Deputy Project Manager \\\hline
DM&Data Management \\\hline
DMCCB&DM Change Control Board \\\hline
DMIS&DM Interface Scientist \\\hline
DMLT&DM Leadership Team \\\hline
DMPM&Data Management Project Manager \\\hline
DMSR&DM System Requirements \\\hline
DMSS&DM Subsystem Scientist \\\hline
DMTN&DM Technical Note \\\hline
DOC&Department of Commerce (USA) \\\hline
DPC&Data Processing Centre \\\hline
DRP&Data Release Production \\\hline
ICD&Interface Control Document \\\hline
IS&Interface Scientist \\\hline
IVOA&International Virtual-Observatory Alliance \\\hline
JIRA&issue tracking product (not an acronym, but a truncation of Gojira, the Japanese name for Godzilla) \\\hline
LCR&LSST Change Request \\\hline
LDM&Light Data Management \\\hline
LPM&LSST Project Management (Document Handle) \\\hline
LSE&LSST System Engineering (Document Handle) \\\hline
LSST&Large-aperture Synoptic Survey Telescope \\\hline
LaTeX&(Leslie) Lamport TeX (document markup language and document preparation system) \\\hline
NCSA&National Center for Supercomputing Applications \\\hline
OCS&Observatory Control System \\\hline
OSS&Operations Support System \\\hline
PDF&Portable Document Format \\\hline
PM&Project Manager \\\hline
PMCS&Project Management Control System \\\hline
PS&Project Scientist \\\hline
PST&Project Science Team \\\hline
QA&Quality Assurance \\\hline
RFC&Request for Comments \\\hline
SA&Science Alert(s) \\\hline
SAT&Science Archives Team (at ESAC) \\\hline
SEMP&System Engineering Management Plan \\\hline
SS&Subsystem Scientist \\\hline
SST&Space Surveillance Telescope \\\hline
SUI&Science User Interface \\\hline
TCT&Technical Control Team (Obsolete - now DMCCB) \\\hline
US&United States \\\hline
WBS&Work Breakdown Structure \\\hline
\end{longtable} 
 % generated by the acronyms.csh (GaiaTools)

%Make sure lsst-texmf/bin/generateAcronyms.py is in your path
\printglossaries



\end{document}
