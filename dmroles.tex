\section{Roles in Data Management}

There are many roles listed in \figref{fig:dmorg}, this section enumerates responsibilities going with those roles. 


\subsection{DM Project Manager (DMPM)\label{role:dmpm}}

The DM Project Manager is responsible for the efficient coordination of all LSST activities and responsibilities assigned to the Data Management Subsystem. The DM Project Manager has the responsibility of establishing the organization, resources, and work assignments to provide DM solutions.  The DM Project Manager, serves as the DM representative in the LSST Project Office and in that role is responsible for presenting DM initiative status and submitting new DM initiatives for approval consideration. Ultimately, the DM Project Manager, in conjunction with his / her peer Project Managers (Telescope, Camera), is responsible for delivering an integrated LSST system. The DM Project Manager reports to the LSST Project Manager. Specific responsibilities include:

\begin{itemize}
\item Manage the overall DM System
\item Define scope and funding for DM System 
\item Develop and implement the DM project management and control process, including earned value management
\item Approve the DM Work Breakdown Structure (WBS), budgets and resource estimates
\item Approve or execute as appropriate all DM outsourcing contracts 
\item Convene and/or participate in all DM reviews
\item Co-Chair the DM Leadership Team
\end{itemize}

\subsection{DM Deputy Project Manager (DDMPM) \label{role:dmdpm}}
The DM Project Manager will keep the Deputy Project Manager informed of all DM situations such that the deputy may effectively act in place of the Project Manager when absent. 
The PM and deputy will work together on the general management of DM and any specific PM tasks may be delegated to the deputy as needed and agreed. In the absence of the PM the deputy carries full authority and decision making powers of the PM. 

\subsection{DM Subsystem Scientist (DMSS) \label{role:dmps} }
The DM Project Scientist has ultimate responsibility for ensuring DM initiatives provide solutions that meet the overall LSST scientific and technical requirements.  The DM Project Scientist must ensure correct specification of DM Scientific Requirements and proper translation of those requirements into derived information technology requirements and ultimately, into implemented solutions.  The DM Project Scientist must ensure that the DM subsystem is properly scoped and integrated within the overall LSST system.  The DM Project Scientist is also a member of the LSST Project Science Team (PST) and reports to the LSST Director. Specific responsibilities include:

\begin{itemize}
\item Responsible for the science deliverables of the DM System
\item Set requirements for the DM that:
\begin{itemize}
\item Ensure that the design and operational flow of the data products meet the needs of the science community
\item Ensure that the quality requirements of the data products will be / are being met by the DMS, with a particular emphasis on choice of appropriate application algorithms
\end{itemize}
\item Set requirements for and assess/validate results of Data Challenges and other precursor experiments
\item Set requirements and assess/validate results for Data Releases
\item Convene and/or participate in all DM reviews
\item Co-Chair the DM Leadership Team and Science/Architecture Team
\end{itemize}

\subsection{ Project Controller/Scheduler \label{role:pcon}}
Keep AGILE plan in sync with the overall LSST planning (primavera), track milestones from TCAMS \secref{role:tcam}. 
Help TCAMS with building the plan from the milestones tracking dependencies and keeping it up to date. 

Help set up sprint - points available (start/end day, account for holidays etc.) 
Bug team in general about story status in sprints and their tracking status (points spent).

Create reports and gantt charts for the DM Project Manager as needed \secref{role:dmpm}
\subsection{Technical Control/Account Manager (TCAM) \label{role:tcam} }
Accountable for planning and execution in their area. Reporting to the DM Project Manager \secref{role:dmpm}. In AGILE could also be seen as the SCRUM Master for the local team.

\subsection{ Product Owner \label{role:prodo}}
The product owner, aka. the X scientist (where X is the product e.g. Alerts Production), is responsible for the product quality and acceptance. 
The product owner should sign off on the requirements to be fulfilled in every delivery and therefore also on any descopes or enhancements. 
The Product owner should define tests which can be run to prove a delivery meets the requirements due for that product. 

\subsection{Pipeline Scientist \label{role:pipe}}
Several DM products come together to form the LSST pipeline. The Pipeline Scientist is the product owner for the overall pipeline. 
The Pipeline Scientist should :
\begin{itemize}
\item  Provide guidance and test criteria for the full pipeline including how QA is done on the products.  
\item Keep the big picture of where the codes are going in view. Predominantly the algorithms, but also the implementation and architecture (as part of the System Engineering Team \secref{sect:sysengt}).


\item Advise on how we should attack algorithmic problems,  
providing continuing advice to subsystem product owners as we try new things. 

\item Advise on calibration issues, provide understanding of the detectors from a DM point of view. 

\item Advise on the overall (scientific) performance of the system, and how we'll test it.  Thinking about all the small things that we have to get right to make the overall system good.



\end{itemize}


\subsection{System Engineer \label{role:sysengineer}}
With the system engineering team \secref{sect:sysengt} 
 the System engineer 
 owns the DM entries in the risk register and is generally in charge of the {\em process} of building DM products. 

The System engineer  
is responsible for the requirements work:
\begin{itemize}
\item E.g., updating the DMSR, OSS, LSR (including traceability)
\item Ensure we’re appropriately modelling and recording information about the system (e.g.,
    drawings, design documents, etc.)
\item Overseeing work on ICDs, lower level requirements documents, etc.
\item Ensuring we have a solid verification plans/standards across the board in DM
\end{itemize}

The System Engineer is responsible for the process to define \& maintain DM interfaces
\begin{itemize}
\item Defining standards for and ensuring internal interfaces are identified and worked out
\item Direct Interface Scientist's work on external ICDs
\end{itemize}

The System Engineer shall Chair the DM Configuration Control Board \secref{sect:tct}
\begin{itemize}
\item Organise DMCCB  processes so our change control process runs smoothly
\item Shepherd RFCs through change control
\item Monitor + Flag RFCs requiring DMCCB  attention
\item Call up meetings, make sure decisions are made, and recorded
\end{itemize}

The System Engineer represents DM on the LSST CCB.

\subsection{DM Interface Scientist (DMIS) \label{role:dmis}}
Look after all internal and external DM interfaces - including defining tests for them. 

\subsection{Requirements Engineer \label{role:reqeng}}
With the system engineering team (\secref{sect:sysengt}) and in close coordination with the software architect (\secref{role:softarc}) and the system engineer (\secref{role:sysengineer}) looks after the baseline requirements for DM.. 




\subsection{Software Architect \label{role:softarc}}
The software architect looks after the software we are building. How does it all fit together are their techniques/technologies we should be using. How can we minimise dependencies. 

With the \secref{role:sysengineer} the Software architect should also agree how to track requirements to code and verify requirements are i.e. are hooks required in the code ?

\subsection{Operations Architect \label{role:opsarc}}
{\em Margaret  or Don perhaps some text here .. }
The DM Operations  Architect is responsible for ensuring that all elements of the DM systems, including operations teams, infrastructure, middle ware, applications, and interfaces, 
all come together to form an operable system. 
Specific responsibilities include:
\begin{itemize}
\item Setting up and coordinating  Operations Rehearsals
\item Ensuring Readiness of procedures and personnel for Operations
\item Set standards for operations e..g procedure handling and operator logging
\item Participate in stakeholder and end user coordination and approval processes and reviews
\item Member of the LSST System Engineering Team
\end{itemize}


\section{Lead Institution Senior Positions}
Each Lead Institution has a Project Manager and Scientific/Engineering Lead, who jointly have overall end product responsibility for a broad area of DM work, typically a Work Breakdown Structure (WBS) Level 2 element. They are supervisors of the team at that institution.  Their roles and responsibilities are similar to the DM Project Manager, DM Project Scientist, and DM System Architect, and DM QA and Test Lead, but within the scope of work assigned to that institution.  These leaders are required to acknowledge and implement direction from the DM leadership in all matters pertaining to the DM project.  The DM Project Manager and DM Project Scientist have direct input into the performance appraisals of the Institution Project Manager and Scientific/Engineering Lead. 
The lead institutions are covered  in \secref{sect:leadtutes}.

%\subsection{ \label{role:}}
